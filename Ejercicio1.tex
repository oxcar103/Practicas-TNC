	$d1$, $d2$, $d3$, $d4$, $d5$, $d6$, $d7$ y $d8$ serán los valores de los dígitos de tu DNI o pasaporte, donde $d1$
	es tu primer dígito. Si te faltan dígitos, añades ceros. Así, $d1d2d3d4d5d6d7d8$ será un número de 8 cifras.
	
	Dado tu número de 8 cifras, $n$:
	\begin{enumerate}
		\item Con el algoritmo de la exponencialización rápida, calcula sus $a-sucesiones$ para los 5 primeros primos.
		\item ¿Qué dice el test de Solovay-Strassen para esas 5 bases? ¿Con qué probabilidad tu $n$ es primo?
	\end{enumerate}

\section*{Solución}
	Lo primero que calcularemos es $n$, que en mi caso sería $n = 26050919 =_b 0001 1000 1101 1000 0001 0110 0111$
	
	\begin{enumerate}
		\item El algoritmo que he utilizado es el de exponencialización rápida de izquierda a derecha modificado
		para que en cada iteración $i$ calcule $a^{exp_i} \mod{n}$ en lugar de $a^{exp_i}$ para poder tratar el
		número pues de otro modo sería intratable. El algoritmo es el siguiente:
		\begin{algorithm}[H]
			\begin{algorithmic}[1]
				\REQUIRE \ \\
					\texttt{$a$}, base \\
					\texttt{$b =_b e_je_{j-1}e_{j-2}\dots e_2e_1e_0$}, exponente y su expresión en binario \\
					\texttt{$c$}, módulo\\ \
				\STATE{\texttt{$acu = 1$}}
				\STATE{\texttt{$exp = 0$}}
				\STATE{\texttt{$i = j$; $j$ es el número de dígitos en binario de $b$}}
				\WHILE{\texttt{$i > 0$}}
					\STATE{\texttt{$acu \equiv acu^2 \mod{c}$ }}
					\STATE{\texttt{$exp = 2 \cdot exp$}}
					\IF{\texttt{$e_i = 1$}}
						\STATE{\texttt{$acu \equiv acu \cdot a \mod{c}$}}
						\STATE{\texttt{$exp = exp + 1$}}
					\ENDIF
					\STATE{\texttt{$i = i - 1$}}
				\ENDWHILE
				\PRINT{\texttt{$acu$, $exp$}}
			\end{algorithmic}
			\caption{Exponencialización rápida de izquierda a derecha}
			\label{Fast-exp}
		\end{algorithm}
		
		Dado que $\displaystyle \frac{n-1}{2} = 13025459$ es impar, las $a-sucesiones$ tendrán sólo 2 términos.
		
		\begin{itemize}
			\item Para $a = 2$ tenemos:\\
				$2^{13025459} \equiv 3764029 \mod 26050919$\\
				$2^{26050918} \equiv 17811015 \mod 26050919$\\
				
			\item Para $a = 3$ tenemos:\\
				$3^{13025459} \equiv 21679447 \mod 26050919$\\
				$3^{26050918} \equiv 11610658 \mod 26050919$\\
				
			\item Para $a = 5$ tenemos:\\
				$5^{13025459} \equiv 6820151 \mod 26050919$\\
				$5^{26050918} \equiv 22769921 \mod 26050919$\\
				
			\item Para $a = 7$ tenemos:\\
				$7^{13025459} \equiv 16034947 \mod 26050919$\\
				$7^{26050918} \equiv 2720332 \mod 26050919$\\
				
			\item Para $a = 11$ tenemos:\\
				$11^{13025459} \equiv 14340865 \mod 26050919$\\
				$11^{26050918} \equiv 22153099 \mod 26050919$\\
				
			Claramente, $n$ es compuesto. De hecho, no engaña a ningún $a$ primo tomado.
		\end{itemize}
		
		\item Para realizar el test de Solovay-Strassen, es necesario comprobar si se cumple $\displaystyle \left(
		\frac{a}{n} \right) \equiv a^{\frac{n-1}{2}} \mod n$. Para la parte de la derecha usaremos los resultados del
		apartado anterior mientras que para la parte izquierda utilizaremos el algoritmo de cálculo del símbolo
		de Jacobi\footnote{Este valor coincide con el símbolo de Legendre cuando $n$ es un primo impar.}. El
		algoritmo es el siguiente:
		\begin{algorithm}[H]
			\begin{algorithmic}[1]
				\REQUIRE \ \\
					\texttt{$a$}, elemento \\
					\texttt{$n$}, base \\ \
				\STATE{\texttt{$t = 1$}}
				\STATE{\texttt{$m = |n|$}}
				\STATE{\texttt{$b \equiv a \mod m$}}
				\WHILE{\texttt{$b != 0$}}
					\WHILE{\texttt{$b$ par}}
						\STATE{\texttt{$\displaystyle b = \frac{b}{2}$}}
						\IF{\texttt{$m \mod 8 \in \{3, 5\}$}}
							\STATE{\texttt{$t = -t$}}
						\ENDIF
					\ENDWHILE
					\STATE{\texttt{$(b, m) = (m, b)$}}
					\IF{\texttt{$b \equiv m \equiv 3 \mod 4$}}
						\STATE{\texttt{$t = -t$}}
					\ENDIF
					\STATE{\texttt{$b \equiv b \mod{m}$}}
				\ENDWHILE
				\IF{\texttt{$m = 1$}}
					\PRINT{\texttt{$t$}}
				\ELSE
					\PRINT{\texttt{$0$}}
				\ENDIF
			\end{algorithmic}
			\caption{Símbolo de Jacobi}
			\label{Jac-symbol}
		\end{algorithm}
		
		Ahora, vayamos a los resultados de nuestro $n$ particular:
		
		\begin{itemize}
			\item Para $a = 2$ tenemos usando el algoritmo:
			$$\left(\frac{2}{26050919} \right) = \left(\frac{2}{26050919} \right) \cdot \left(\frac{1}{26050919}
			\right) = 1 \cdot \left(\frac{1}{26050919} \right) = 1$$
			Recordemos que $ a^{\frac{n-1}{2}} = 2^{13025459} \equiv 3764029 \mod 26050919$.
			
			Por tanto, \textbf{NO} pasa el test para $a = 2$.
			
			\item Para $a = 3$ tenemos usando el algoritmo:
			$$\left(\frac{3}{26050919} \right) = -\left(\frac{26050919}{3} \right) = -\left(\frac{2}{3}\right) =
			-(-1) = 1$$
			Recordemos que $ a^{\frac{n-1}{2}} = 3^{13025459} \equiv 21679447 \mod 26050919$.
			
			Por tanto, \textbf{NO} pasa el test para $a = 3$.
			
			\item Para $a = 5$ tenemos usando el algoritmo:
			$$\left(\frac{5}{26050919} \right) = \left(\frac{26050919}{5} \right) =  \left(\frac{4}{5} \right) =
			\left(\frac{2^2}{5} \right) = \left(\frac{2}{5} \right)^2 = (-1)^2 = 1$$
			Recordemos que $ a^{\frac{n-1}{2}} = 5^{13025459} \equiv 6820151 \mod 26050919$.
			
			Por tanto, \textbf{NO} pasa el test para $a = 5$.
			
			\item Para $a = 7$ tenemos usando el algoritmo:
			$$\left(\frac{7}{26050919} \right) = -\left(\frac{26050919}{7} \right) = -\left(\frac{6}{7} \right) =
			-\left(\frac{2}{7} \right) \cdot \left(\frac{3}{7} \right) = -\left(-\left(\frac{7}{3} \right) \right)
			= \left(\frac{1}{3} \right) = 1$$
			Recordemos que $ a^{\frac{n-1}{2}} = 7^{13025459} \equiv 16034947 \mod 26050919$.
			
			Por tanto, \textbf{NO} pasa el test para $a = 7$.
			
			\item Para $a = 11$ tenemos usando el algoritmo:
			$$\left(\frac{11}{26050919} \right) = -\left(\frac{26050919}{11} \right) = -\left(\frac{4}{11} \right) = -1$$
			Recordemos que $ a^{\frac{n-1}{2}} = 11^{13025459} \equiv 14340865 \mod 26050919$.
			
			Por tanto, \textbf{NO} pasa el test para $a = 11$.		
		\end{itemize}
		
		Para concluir que nuestro número $n$ es compuesto nos basta con ver que falla una única de las pruebas
		anteriores y las falla todas. Por tanto, $n$ es compuesto con probabilidad de 1\footnote{En caso de que
		hubiera pasado las pruebas, nuestro número sería primo con una probabilidad mayor que $1-2^{-k} =
		1-2^{-5} = 0.96875$.}.
		
	\end{enumerate}
