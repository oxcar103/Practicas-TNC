	Los parámetros  de un criptosistema de ElGamal son $p = 211$ y $g = 3$, es decir, el cristosistema está
	diseñado en el cuerpo $\mathbb{F}_{211} = \mathbb{Z}_{211}$ y tomamos $g = 3$ como generador de
	$\mathbb{F}^*_{211}$. La clave pública empleada es $3^a = 109  \mod 211$. Descifra el criptograma
	$(154, \textit{dni} \mod 211)$, donde \textit{dni} es nuestro número de DNI. Para calcular los logaritmos
	discretos necesarios emplea dos de los métodos descritos en la teoría.
\section*{Solución}
	Los parámetros públicos que utilizará ElGamal son:
	\begin{itemize}
		\item $p = 211$
		\item $g = 3$
		\item $g^a = 109$
	\end{itemize}
	
	Para el cifrado ElGamal nos basta tomar el mensaje $m$ y un valor aleatorio $1 < k < p-1$ y devolver como
	criptograma $c = \left(g^k \mod p, m \cdot \left(g^a\right)^k \mod p\right)$.
	
	Para el descifrado ElGamal, se utiliza el criptograma $c = \left(x, y\right)$ y el mensaje se obtiene como
	$m = y \cdot x^{-a}$.
	
	Fácil, ¿no?
	
	Pues no, porque resulta que en nuestro ejemplo particular, actuamos como el atacante, y por tanto no conocemos
	el valor de $a$, tenemos que descubrirlo a partir de $g^a$. Para ello, veremos dos métodos ampliamente usados.
	
