	Los parámetros  de un criptosistema de ElGamal son $p = 211$ y $g = 3$, es decir, el cristosistema está
	diseñado en el cuerpo $\mathbb{F}_{211} = \mathbb{Z}_{211}$ y tomamos $g = 3$ como generador de
	$\mathbb{F}^*_{211}$. La clave pública empleada es $3^a = 109  \mod 211$. Descifra el criptograma
	$(154, \textit{dni} \mod 211)$, donde \textit{dni} es nuestro número de DNI. Para calcular los logaritmos
	discretos necesarios emplea dos de los métodos descritos en la teoría.
\section*{Solución}
	Los parámetros públicos que utilizará ElGamal son:
	\begin{itemize}
		\item $p = 211$
		\item $g = 3$
		\item $g^a = 109$
	\end{itemize}
	
	Para el cifrado ElGamal nos basta tomar el mensaje $m$ y un valor aleatorio $1 < k < p-1$ y devolver como
	criptograma $c = \left(g^k \mod p, m \cdot \left(g^a\right)^k \mod p\right)$.
	
	Para el descifrado ElGamal, se utiliza el criptograma $c = \left(x, y\right)$ y el mensaje se obtiene como
	$m = y \cdot x^{-a}$.
	
	Fácil, ¿no?
	
	Pues no, porque resulta que en nuestro ejemplo particular, actuamos como el atacante, y por tanto no conocemos
	el valor de $a$, tenemos que descubrirlo a partir de $g^a$. Para ello, veremos dos métodos ampliamente usados.
	
	El primero será el algoritmo Paso de bebé - Paso de Gigante\footnote{Nótese además que la tabla del Paso de
	bebé y el Paso de Gigante se pueden reutilizar si se mantiene tanto el cuerpo como el generador.}:
	
	\begin{algorithm}[H]
		\begin{algorithmic}[1]
			\REQUIRE \ \\
				\texttt{$b$}, generador\\
				\texttt{$n$}, orden del generador\\
				\texttt{$h = b^a$}, elemento al que calcular el logaritmo\\ \
			\STATE{\texttt{$r = \left\lceil\sqrt{n}\right\rceil$}}
			\STATE{\texttt{$BABY = \left[\ \right]$}}
			\FOR{\texttt{$0 \leq i \leq r-1$}}
				\STATE{\texttt{$BABY\left[b^i\right] = i$}, almacenada así por la futura forma de acceso a ella.}
			\ENDFOR
			\STATE{\texttt{$GIANT = b^{-r}$}}
			
			\STATE{\texttt{$h_0 = h$}}
			\FOR{\texttt{$0 \leq i \leq r-1$}}
				\IF{\texttt{$h_i \in BABY$}, damos pasos de bebé.}
					\PRINT{\texttt{$\log_b h =BABY[h_i] + i \cdot r$}}
				\ELSE
					\STATE{\texttt{$h_{i+1} = h_i \cdot GIANT$}, damos un paso de Gigante.}
				\ENDIF
			\ENDFOR
			
			\PRINT{\texttt{No existe el logaritmo}}
		\end{algorithmic}
		\caption{Algoritmo Paso de bebé - Paso de Gigante.}
		\label{BabyGiant}
	\end{algorithm}
	
	En nuestro caso particular, $GIANT = 3^{-15} = 67$ y la tabla $BABY$ es:
	\begin{center}
	\begin{tabular}{ | r | c  c  c  c  c  c  c  c  c  c  c  c  c  c  c |}
		\hline
		$b^i$   &  1  &  3  &  9  &  27 &  81 &  32 &  96 &  77 &  20 &  60 & 180 & 118 & 143 &  7  &  21 \\
		\hline
		$i$     &  0  &  1  &  2  &  3  &  4  &  5  &  6  &  7  &  8  &  9  &  10 &  11 &  12 &  13 &  14 \\
		\hline
	\end{tabular}
	\end{center}
	
	Y los valores por los que pasamos son:
	\begin{itemize}
		\item $h_0 = 109$
		\item $h_1 = 129$
		\item $h_2 = 203$
		\item $h_3 = 97$
		\item $h_4 = 169$
		\item $h_5 = 140$
		\item $h_6 = 96$
	\end{itemize}
	
	En conclusión, vemos que $h_6 = b^6 \Rightarrow h = b^{6 + 6 \cdot 15} = b^{96} \Rightarrow \log_b h = 96$.