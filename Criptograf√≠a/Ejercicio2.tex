	\begin{enumerate}
		\item Supongamos que tenemos almacenada una lista de todos los primos menores que $\sqrt{n}$. Estudia la
		complejidad estimada en determinar si $n$ es primo dividiendo entre todos los primos de la lista anterior.
		Recordemos que el teorema del número primo dice:
		$$\lim\limits_{n \rightarrow \infty} \frac{\pi (n)}{\frac{n}{\log n}} = 1$$
		donde $\pi (n)$ es el número de primos menores que $n$.
		\item Calcula una estimación del tiempo necesario para saber si $n$ es divisible por un primo menor o
		igual que $m$, suponiendo que tenemos una lista con todos los primos menores o iguales que $m$.
		\item Calcula la complejidad del producto de matrices con coeficientes en $\mathbb{Z}_m$ en función del
		tamaño de las mismas.
	\end{enumerate}
\section*{Solución}
	\begin{enumerate}
		\item Como vimos en un ejercicio anterior, sabemos que $\lim\limits_{n \rightarrow \infty} \frac{f(n)}
		{g(n)} = \lambda \in \mathbb{R} \Rightarrow f = \mathcal{O}(g)$. Por tanto, sabemos que $\pi (n) =
		\mathcal{O}\left(\frac{n}{\log n}\right)$. Además, sabemos que el coste de calcular el cociente y el
		resto de la división $\frac{p}{q} = \mathcal{O}((\log p) \cdot (\log q))$.
		
		Por tanto, comprobar si un número es primo requiere aproximadamente de $\frac{\sqrt{n}}{\log \sqrt{n}}$
		divisiones de dividendo menor o igual que $\sqrt{n}$, es decir, la comprobación de primalidad es de orden
		$\mathcal{O}\left(\frac{\sqrt{n}}{\log \sqrt{n}} \cdot (\log n) \cdot \left(\log \sqrt{n}\right)\right)
		= \mathcal{O} \left(\sqrt{n} \cdot \log n\right)$.
		
		\item En este caso, el cálculo del orden de la comprobación de primalidad requiere de una apreciación
		previa y es que el sabemos que si $n$ no es divisible por un primo menor que $\sqrt{n}$, es primo.
		Por tanto, debemos dividir en 2 casos:
		\begin{enumerate}
			\item $m > \sqrt{n}$: en este caso, tendremos el mismo orden que en el apartado anterior, es decir,
			$\mathcal{O} \left(\sqrt{n} \cdot \log n\right)$
			\item $m < \sqrt{n}$: en este caso, tendremos que hacer $\pi (m)$ divisiones con dividendo menor o
			igual que $m$, esto es, el orden aproximadamente es $\mathcal{O}\left(\frac{m}{\log m} \cdot
			(\log n) \cdot (\log m)\right) = \mathcal{O} \left(m \cdot \log n\right)$
		\end{enumerate}
		
		\item Para calcular el orden del producto de matrices, tenemos que tener en cuenta qué orden tienen
		las matrices que tomamos como factores. En nuestro caso, tomaremos como factores las matrices genéricas
		$A$ y $B$ de orden $i \times j$ y $j \times k$, respectivamente, y $C = A \cdot B$ con orden $i \times k$
		como matriz producto.
		
		Para calcular $c_{ik}$, haremos $\displaystyle c_{ik} = \sum_{l=1}^{j} a_{il} \cdot b_{lk}$ que requiere
		de $j$ multiplicaciones y $j-1$ sumas del cuerpo $\mathbb{Z}_m$. Además, tenemos que calcular todos los
		términos de $C$, que tiene $i \cdot k$ elementos. Por tanto, tenemos que:
		$$\mathcal{O}(C) = \left((j-1) \cdot \mathcal{O}(\text{Suma en }\mathbb{Z}_m) +j \cdot
		\mathcal{O}(\text{Producto en } \mathbb{Z}_m)\right) \cdot i \cdot k$$
		
		Sabemos además que $\mathcal{O}(\text{Suma en }\mathbb{Z}_m) = \mathcal{O}(\log m)$ y que $\mathcal{O}
		(\text{Producto en } \mathbb{Z}_m) = \mathcal{O}((\log m)^2)$. Por tanto, concluimos que el orden del
		producto de matrices de orden $i \times j$ y $j \times k$ en $\mathbb{Z}_m$ es
		$\left((j-1) \cdot \mathcal{O}(\log m) +j \cdot \mathcal{O}\left((\log m)^2\right)\right) \cdot i \cdot k =
		\mathcal{O}\left(i \cdot j \cdot k \cdot (\log m)^2\right)$.
	\end{enumerate}
