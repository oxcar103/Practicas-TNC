	Sea $p$ el siguiente número primo mayor estricto que tu DNI y sea la curva elíptica $E: y^2 = x^3 + 4x$.
	\begin{enumerate}
		\item Calcula el número de puntos módulo $p$, que denotaremos $M_p(E)$.
		\item Dado el punto $P = (2,4)$, calcula el orden en el grupo abeliano de la curva elíptica $E$.
	\end{enumerate}
	
\section*{Solución}
	\begin{enumerate}
		\item Para calcular el número de puntos módulos $p$ podríamos basarnos en un resultado visto en teoría
		en el cuál si $p$ es un primo impar, se tiene una expresión para calcularlo:
		$$\displaystyle M_p(E) = p + 1 + \sum_{x=0}^{p-1} \left( \frac{x^3 + ax^2 + bx + c}{p} \right)$$
		
		Sin embargo, este resultado, aunque eficaz, no es eficiente dado el gran número de veces que es necesario
		aplicar el símbolo de Jacobi.
		
		Por ello, nos aprovecharemos de otro resultado que tiene como condiciones:
		\begin{itemize}
			\item $q \equiv_4 3$ primo.
			\item La curva elíptica $F$ sea de la forma $y^2 = x^3 + dx$
			\item $d \in \mathbb{Z}$ no múltiplo de $q$
		\end{itemize}
		Entonces, se tiene que $M_q(F) = q+1$.

		Claramente, nuestro $p$, nuestra $E$ y nuestro $b$ cumplen las condiciones, por lo que tenemos $M_p(E)
		= p+1 = 26050968$.
		
		\item Para calcular el orden de $P$, nos basta con sumarlo sucesivamente consigo mismo hasta obtener
		como resultado el punto del infinito: $<0, 1, 0>$
		
		Previamente, tenemos que tener en cuenta algunos detalles:
		\begin{itemize}
			\item Si tenemos $Q = (x_q, y_q)$ y $R = (x_r, y_r)$, entonces:
			$$S = Q + R = (x_s, y_s) \equiv_p (m^2 - a - x_q - x_r, -y_q - m(x_s - x_q)) \quad \text{con }
			m=\frac{y_r - y_q}{x_r - x_q}$$
			\item Si tenemos $Q = (x, y)$ y $-Q = (x, -y)$, entonces:
			$$S = Q + (-Q) = (x_s, y_s) \equiv_p <0, 1, 0>$$
			\item Si tenemos $Q = (x, y)$, entonces:
			$$S = Q + Q = (x_s, y_s) \equiv_p (m^2 - a - 2x, -y - m(x_s - x)) \quad \text{con }
			m=\frac{3x^2 + 2ax + b}{2y}$$
		\end{itemize}
		
		Entonces, para nuestro $P = (2, 4)$ tenemos que:
		\begin{enumerate}
			\item $\displaystyle m=\frac{3x^2 + 2ax + b}{2y} = \frac{3 \cdot 2^2 + 2 \cdot 0 \cdot 2 + 4}
			{2 \cdot 4} = \frac{12 + 0 + 4}{8} = 2$
			\item $x_s \equiv_p m^2 - a - 2x = 2^2 - 0 - 2 \cdot 2 = 0$
			\item $y_s \equiv_p -y - m(x_s - x) = -4 - 2(0 - 2) = 0$
		\end{enumerate}
		
		Por tanto, se tiene que $2P = P + P = (0, 0)$.
		
		En teoría, deberíamos seguir hasta llegar al punto del infinito pero, en este caso, podemos ver
		que $2P = (0, 0)$ tiene orden 2 pues es su propio inverso. Por tanto, como $2P$ tiene orden 2,
		se tiene que $P$ tiene orden 4.
	\end{enumerate}