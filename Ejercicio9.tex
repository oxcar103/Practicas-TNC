	Sea $p$ el siguiente número primo mayor estricto que tu DNI y sea la curva elíptica $E: y^2 = x^3 + 4x$.
	\begin{enumerate}
		\item Calcula el número de puntos módulo $p$, que denotaremos $M_p(E)$.
		\item Dado el punto $P = (2,4)$, calcula el orden en el grupo abeliano de la curva elíptica $E$.
	\end{enumerate}
	
\section*{Solución}
	\begin{enumerate}
		\item Para calcular el número de puntos módulos $p$ podríamos basarnos en un resultado visto en teoría
		en el cuál si $p$ es un primo impar, se tiene una expresión para calcularlo:
		$$\displaystyle M_p(E) = p + 1 + \sum_{x=0}^{p-1} \left( \frac{x^3 + ax^2 + bx + c}{p} \right)$$
		
		Sin embargo, este resultado, aunque eficaz, no es eficiente dado el gran número de veces que es necesario
		aplicar el símbolo de Jacobi.
		
		Por ello, nos aprovecharemos de otro resultado que tiene como condiciones:
		\begin{itemize}
			\item $q \equiv_4 3$ primo.
			\item La curva elíptica $F$ sea de la forma $y^2 = x^3 + bx$
			\item $b \in \mathbb{Z}$ no múltiplo de $q$
		\end{itemize}
		Entonces, se tiene que $M_q(F) = q+1$.

		Claramente, nuestro $p$, nuestra $E$ y nuestro $b$ cumplen las condiciones, por lo que tenemos $M_p(E)
		= p+1 = 26050968$.
		
		\item 
	\end{enumerate}