	Sea $n$ el siguiente número primo\footnote{Aunque conocemos este hecho, dado que la utilidad de este ejercicio
	es conocer y poner a prueba nuestro conocimiento sobre tests de primalidad, lo ignoraremos y trataremos a $n$
	como un número que queremos ver si es primo.} mayor estricto que tu DNI, así $n$ tendrá 8 cifras decimales.

	\begin{enumerate}
		\item Encuentra $Q$ el primer natural mayor que 1 que satisface $d=1-4Q$ no es cuadrado perfecto, su símbolo de
		Jacobi es $\displaystyle \left(\frac{d}{n}\right) = -1$ y que además la sucesión de Lucas asociada certifica
		la primalidad de tu $n$.
		\item Encuentra el primer natural mayor que 1 que sea primitivo y, por tanto, certifique la primalidad de tu $n$.
	\end{enumerate}

\section*{Solución}
	\begin{enumerate}
		\item En este apartado, usaremos los conocimientos que ya tenemos del cálculo del símbolo de Jacobi (algoritmo
		\ref{Jac-symbol}) y del cálculo rápido de términos de la sucesión de Lucas (algoritmo \ref{Lucas-Suc}) en el
		siguiente algoritmo:
		
		\begin{algorithm}[H]
		\begin{algorithmic}[1]
			\REQUIRE \ \\
				\texttt{$P$}, parámetro de definición de $\alpha$ \\
				\texttt{$n$}, número al que hacer el test \\
			\STATE{\texttt{$Q = 2$, valor inicial de la búsqueda}}
			\STATE{\texttt{$Encontrado = \FALSE$}}
			\WHILE{\texttt{$Encontrado == \FALSE$ \AND $Q < r$}}
				\STATE{\texttt{$Encontrado = \TRUE$}}
				\STATE{\texttt{$d = P^2-4Q$}}
				\STATE{\texttt{$s = Jac\_symbol(d, n)$}}
				\IF{\texttt{$s == 1$}}
					\PRINT{\texttt{$Q$, $s$}}
					\STATE{\texttt{$Encontrado = \FALSE$}}
				\ELSE
					\FOR{\texttt{$p_i \in Prime\_Factors(n+1)$}}
						\IF{\texttt{$U_\frac{n+1}{p_i} \equiv_n 0$}}
							\PRINT{\texttt{$Q$, $U_\frac{n+1}{p_i} \equiv_n 0$}}
							\STATE{\texttt{$Encontrado = \FALSE$}}
						\ENDIF
					\ENDFOR
				\ENDIF
				\IF{\texttt{$Encontrado == \TRUE$}}
					\PRINT{\texttt{$Q$, $d$, $s$}}
					\FOR{\texttt{$p_i \in Prime_Factors(n+1)$}}
						\PRINT{\texttt{$U_\frac{n+1}{p_i} \mod n$}}
					\ENDFOR
				\ELSE
					\STATE{\texttt{$Q += 1$}}
				\ENDIF
			\ENDWHILE
			\PRINT{\texttt{Aceptado $Q$}}
		\end{algorithmic}
		\caption{Test de primalidad usando sucesiones de Lucas.}
		\label{Primarity}
		\end{algorithm}
		
		Tras aplicar el algoritmo a nuestro $n$ \footnote{Recordemos que para este ejercicio, nuestro $n$ era
		distinto, en lugar de ser el DNI, es el siguiente primo, en mi caso, $n = 26050967$. Además, cabe destacar
		que $n+1 = 2^3 \cdot 3^2 \cdot 181 \cdot 1999$ y que:
		\begin{itemize}
			\item $\displaystyle \frac{n+1}{2} = 13025484$
			\item $\displaystyle \frac{n+1}{3} = 8683656$
			\item $\displaystyle \frac{n+1}{181} = 143928$
			\item $\displaystyle \frac{n+1}{1999} = 13032$
		\end{itemize}} obtenemos:
		\begin{itemize}
			\item $Q = 2$ \\
			$U_{13025484} = 0$ \\
			$U_{8683656} = 0$
			
			\item $Q = 3$ \\
			$U_{13025484} = 0$ \\
			$U_{8683656} = 0$
			
			\item $Q = 4$ \\
			$Jacobi\ symbol = 1$
			
			\item $Q = 5$ \\
			$d = -19$ \\
			$Jacobi\ symbol = -1$ \\
			$U_{13025484} \equiv_n 13006485$ \\
			$U_{8683656} \equiv_n 5392704$ \\
			$U_{143928} \equiv_n 18519423$ \\
			$U_{13032} \equiv_n 20065576$
			
			\item Aceptado $Q = 5$
		\end{itemize}

				
			
	\end{enumerate}
