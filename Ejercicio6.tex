	Sea $n$ el siguiente número primo\footnote{Aunque conocemos este hecho, dado que la utilidad de este ejercicio
	es conocer y poner a prueba nuestro conocimiento sobre tests de primalidad, lo ignoraremos y trataremos a $n$
	como un número que queremos ver si es primo.} mayor estricto que tu DNI, así $n$ tendrá 8 cifras decimales.

	\begin{enumerate}
		\item Encuentra $Q$ el primer natural mayor que 1 que satisface $d=1-4Q$ no es cuadrado perfecto, su símbolo de
		Jacobi es $\displaystyle \left(\frac{d}{n}\right) = -1$ y que además la sucesión de Lucas asociada certifica
		la primalidad de tu $n$.
		\item Encuentra el primer natural mayor que 1 que sea primitivo y, por tanto, certifique la primalidad de tu $n$.
	\end{enumerate}

\section*{Solución}
	\begin{enumerate}
		\item \footnote{En este apartado, no se justificarán los pasos a seguir pues ya han sido explicados.}
		Procedemos a comprobar los resultados con cada uno de los posibles valores de $Q$ aunque en mi caso basta
		con probar el 2 como ya veremos:
		
		\begin{itemize}
			\item $P = 1$
			\item $Q = 2$
			\item $d = P^2 - 4Q = -7$
			$$\displaystyle \left(\frac{d}{n}\right) = \left(\frac{-7}{n}\right) = \left(\frac{26050960}{26050967}
			\right)= \left(\frac{13025480}{26050967}\right) = \left(\frac{6512740}{26050967}\right) =
			\left(\frac{3256370}{26050967}\right) = \left(\frac{1628185}{26050967}\right) = \atop
			\displaystyle \left(\frac{26050967}{1628185}\right) = \left(\frac{7}{1628185}\right) =
			\left(\frac{1628185}{7}\right) = \left(\frac{6}{7}\right) = \left(\frac{3}{7}\right) =
			-\left(\frac{7}{3}\right) = -\left(\frac{1}{3}\right) = -1$$
			\item $r = n+1 = 26050968$
			\item $\displaystyle \alpha = \frac{P + \sqrt{d}}{2} = \frac{1 + \sqrt{-7}}{2}$
			\item El polinomio primitivo mínimo de $\alpha$ es $f(x) = x^2 -x +2$
			\item La sucesión de Lucas se define como $\displaystyle \alpha^n = \frac{V_n}{2} + \frac{U_n}{2}\sqrt{d}$ con $U_i, V_i \in \mathbb{Z}$.
			$$\left\lbrace
				V_n = V_{n-1}-2V_{n-2} \qquad \text{con } V_0 = 2, \quad V_1 = P = 1 \atop
				U_n = U_{n-1}-2U_{n-2} \qquad \text{con } U_0 = 0, \quad U_1 = 1
			\right.$$
			\item Si $n$ es primo, tenemos que:
			$$\alpha^r = \alpha^{n-\left(\frac{d}{n}\right)} \equiv_n
			\left\lbrace
				\displaystyle 1 \qquad \text{Si } \left(\frac{d}{n}\right) = 1 \atop
				\displaystyle Q \qquad \text{Si } \left(\frac{d}{n}\right) = -1
			\right.$$
			
			$$U_r \equiv_n 0 \text{ y } V_r \equiv_n 2\alpha^r$$
			
			$$V_{\frac{r}{2}}^2 \equiv_n 2Q\left(1 + Q^{\frac{r}{2}-1}\right) \equiv_n
			\left\lbrace
				\displaystyle 4Q \quad \text{si } \left(\frac{Q}{n}\right) =\footnote{} 1 \Rightarrow V_{\frac{r}{2}} = 2\sqrt{Q} \atop
				\displaystyle 0 \quad \text{si } \left(\frac{Q}{n}\right) = -1 \Rightarrow V_{\frac{r}{2}} = 0
			\right.$$
			
			\footnotetext{Este valor del símbolo de Jacobi asegura la existencia de, al menos, una raíz de Q que
			notaremos $\sqrt{Q}$.}
			
			$$U_{\frac{r}{2}} \equiv_n 0 \quad \text{si } \displaystyle \left(\frac{Q}{n}\right) = 1$$
			
			\item Los resultados obtenidos son:
			\begin{center}
				\begin{tabular}{ | r | c | c |}
					\hline
					i        & $V_i$    & $U_i$ \\
					\hline
					1        & 1        & 1 \\
					3        & 26050962 & 26050966 \\
					6        & 9        & 5 \\
					12       & 26050920 & 45 \\
					24       & 26044984 & 26048852 \\
					49       & 8933705  & 7447951 \\
					99       & 18087249 & 17180450 \\
					198      & 16844930 & 18943141 \\
					397      & 11268102 & 7494412 \\
					795      & 22482324 & 25661603 \\
					1590     & 24654452 & 20686173 \\
					3180     & 15566822 & 17693380 \\
					6360     & 13786398 & 13014252 \\
					12720    & 24958959 & 5018711 \\
					25440    & 22538925 & 25671904 \\
					50880    & 8020730  & 2610045 \\
					101761   & 19654530 & 20974668 \\
					203523   & 16940953 & 24507923 \\
					407046   & 6104124  & 24598449 \\
					814092   & 23323159 & 18481317 \\
					1628185  & 8774083  & 15973888 \\
					3256371  & 0        & 16990013 \\
					6512742  & 1550734  & 0 \\
					13025484 & 5587493  & 0 \\
					26050968 & 4        & 0 \\
					\hline
				\end{tabular}
			\end{center}
					
			\item Dado que $\displaystyle \left(\frac{Q}{n}\right) = \left(\frac{2}{26050967} \right)= 1$, claramente
			se cumplen los pronósticos calculados, por lo que se $n$ sería candidato a primo.
			
			\item Por tanto, $Q = 2$ verifica todas las hipótesis necesarias para este ejercicio.
		\end{itemize}
	\end{enumerate}
