	Considerando nuestro $DNI$, lo separamos en 4 cifras decimales y, si alguna de las dos partes empieza por 0,
	la rotamos a izquierda hasta que no lo sea. En el caso de que esto no sea posible porque una de esas partes
	es 0000, se puede tomar un número de 4 cifras que no empiece por 0 a elección.
	
	Sean $p$ y $q$ los primeros primos mayores o iguales que los bloques anteriores, sea $n = pq$ y $e \geq 11$
	el menor primo no divisor de $\phi(n)$\footnote{Nótese que $\phi$ es la función totiente de Euler.} y sea
	$d \equiv e^{-1} \mod \phi(n)$.
	\begin{enumerate}
		\item Cifra el mensaje $m = \mathrm{0xCAFE}$.
		\item Descifra el criptograma anterior.
		\item Intenta factorizar $n$ mediante el método $P-1$ de Pollard llegando, como máximo\footnote{Spoiler:
		Dado que el número es pequeño, no voy a respetar este límite.}, a $b = 8$.
		\item Intenta factorizar $n$ a partir de $\phi(n)$.
		\item Intenta factorizar $n$ a partir de $d$
	\end{enumerate}
\section*{Solución}
	Lo primero que calculamos son los parámetros que utilizará RSA. En particular, para mi $DNI$ se tiene que:
	\begin{itemize}
		\item $(p,\ q) = (2609,\ 9199)$
		\item $n = 24000191, \quad \phi(n) = 23988384$
		\item $e = 11, \quad d \equiv e^{-1} \mod \phi(n) = 10903811$
	\end{itemize}
	
	\begin{enumerate}
		\item Para el cifrado RSA nos basta con calcular $c = m^e \mod n$, veámoslo en nuestro caso particular:
		$$\mathrm{0xCAFE}^{11} = \mathrm{0xCAFE}^{1011_2} = \left(\left(\left(\mathrm{0xCAFE}^2\right)^2\right)
		\cdot \mathrm{0xCAFE}\right)^2 \cdot \mathrm{0xCAFE} \equiv \mathrm{0xE3CD33} \mod n$$
		
		\item Para el descifrado RSA, utilizaremos $d$ en lugar de $e$, es decir, $m = c^d \mod n$: \\
		\begin{gather*}
			\mathrm{0xE3CD33}^{10903811} = \mathrm{0xE3CD33}^{1010\ 0110\ 0110\ 0001\ 0000\ 0011_2} = \\
			\left(\left( \cdots \left(\left(\left(\mathrm{0xE3CD33}^2\right)^2 \cdot \mathrm{0xE3CD33}
			\right)^2\right)^2 \cdots \right)^2 \cdot \mathrm{0xE3CD33}\right)^2 \cdot \mathrm{0xE3CD33}
			\equiv \mathrm{0xCAFE} \mod n
		\end{gather*}
		
		Adicionalmente, y para mejorar la eficiencia del cálculo podemos calcular $m = c^{d \mod \phi(p)} \mod p$
		y $m = c^{d \mod \phi(q)} \mod q$, y resolver el sistema de congruencias resultante:
		$$\left.\begin{aligned}
		        \mathrm{0xE3CD33}^{10903811 \mod \phi(2609)} &= \mathrm{0xE3CD33}^{2371} &\equiv \mathrm{0x95B} \mod 2609\\
		        \mathrm{0xE3CD33}^{10903811 \mod \phi(9199)} &= \mathrm{0xE3CD33}^{4181} &\equiv \mathrm{0x1753} \mod 9199
		       \end{aligned}
		\right\} = \mathrm{0xCAFE} = m$$
		
		\item Antes de empezar el proceso de factorización, vamos a ver el algoritmo:
		
		\begin{algorithm}[H]
			\begin{algorithmic}[1]
				\REQUIRE \ \\
					\texttt{$n$}, base\\
					\texttt{$b_0$}, caso inicial\\ \
				\STATE{\texttt{$b = b_0$}}
				\STATE{\texttt{$a = 2^{b!} \mod n$}}
				\STATE{\texttt{$g = MCD(a-1, n)$}}
				\WHILE{\texttt{$g == 1$}}
					\STATE{\texttt{$b += 1$}}
					\STATE{\texttt{$a = a^b \mod n$}}
					\STATE{\texttt{$g = MCD(a-1, n)$}}
				\ENDWHILE
				
				\PRINT{\texttt{$(g,\ n/g)$}, factores de $n$}
			\end{algorithmic}
			\caption{Método P-1 de Pollard.}
			\label{PollardP-1}
		\end{algorithm}
		
		En particular, en mi implementación $b_0$ no es un parámetro, está fijado a 2.
		
		Vista la parte teórica, pasamos a los valores particulares de nuestro caso:
		\begin{center}
		\begin{tabular}{ | r | c | c |}
			\hline
			$b$     & $a \mod n$    & GCD \\
			\hline
			2       &  4            &  1  \\
			3       &  64           &  1  \\
			4       &  16777216     &  1  \\
			5       &  9797646      &  1  \\
			6       &  5351708      &  1  \\
			7       &  19657572     &  1  \\
			8       &  2137360      &  1  \\
			9       &  9962730      &  1  \\
			10      &  983434       &  1  \\
			\vdots  &  \vdots       &  \vdots  \\
			70      &  6326632      &  1  \\
			71      &  12973364     &  1  \\
			72      &  424119       &  1  \\
			73      &  5519401      &  9199  \\
			\hline
		\end{tabular}
		\end{center}
		
		Y de aquí deducimos que $\displaystyle 24000191 = 9199 \cdot \frac{24000191}{9199} = 9199 \cdot 2609$.
		
		\item Para explotar este dato adicional, haremos uso de resultados muy básicos:
		\begin{itemize}
			\item $n = pq$ con $p,\ q$ primos $\Rightarrow \phi(n) = (p-1) \cdot (q-1) = pq -p -q +1 = n -(p+q) +1
			\Rightarrow p+q = n+1-\phi(n)$
			\item $(x-a) \cdot (x-b) = x^2 -(a+b) \cdot x + ab$
			\item $\displaystyle a \cdot x^2 + b \cdot x + c = \frac{-b \pm \sqrt{b^2 - 4 \cdot a \cdot c}}{2 \cdot a}$
		\end{itemize}
		
		Por tanto, queda claro que si calculamos $p+q = n+1-\phi(n) = 24000191+1-23988384 = 11808$ y tomamos $2b = p+q =
		2 \cdot 5904$, tenemos que $\Delta = b^2-n = 10857025 \Rightarrow \sqrt{\Delta} = 3295$, por lo que $(p,\ q) =
		\left(b-\sqrt{\Delta},\ b+\sqrt{\Delta}\right) = (2609, 9199)$
		
		\item Si en lugar de $\phi(n)$, conocemos la clave de desencriptado $d$, podemos calcular $n = pq$ basándonos
		en la propiedad de que como $n$ no es primo, la unidad tiene raíces distintas de $1$ y $-1$:
		
		\begin{algorithm}[H]
			\begin{algorithmic}[1]
				\REQUIRE \ \\
					\texttt{$n$}, base\\
					\texttt{$e$}, clave de cifrado\\
					\texttt{$d$}, clave de descifrado\\ \
				\STATE{\texttt{$g = 1$}}
				\WHILE{\texttt{$g \in \{1, n\}$}}
					\STATE{\texttt{$k = e \cdot d -1$}}
					\STATE{\texttt{$a = rand(2, n-1)$}}
					\STATE{\texttt{$g = MCD(a, n)$}, si el elemento escogido no es coprimo, ya tenemos un factor.}
					\WHILE{\texttt{$g \in \{1, n\}$ \AND $k \equiv 0 \mod 2$}}
						\STATE{\texttt{$\displaystyle k = \frac{k}{2}$}}
						\STATE{\texttt{$a = a^k \mod n$}}
						\STATE{\texttt{$g = MCD(a-1, n)$}}
					\ENDWHILE
				\ENDWHILE
				
				\PRINT{\texttt{$(g,\ n/g)$}, factores de $n$}
			\end{algorithmic}
			\caption{Factorización de $n$ conociendo la clave $d$ de descifrado.}
			\label{AtaqueD}
		\end{algorithm}
		
		Una vez visto el algoritmo, pasamos a los resultados prácticos:
		\begin{table}[H]
		\centering
		\begin{tabular}{| c | c | c | c |}
		\hline
		$a$         & $k$             & $a^k \mod n$        & GCD       \\ \hline
		12601639    & -               & -                   & 1         \\ \cline{2-4}
		            & 59970960        & 1                   & 24000191  \\ 
		            & 29985480        & 1                   & 24000191  \\ 
		            & 14992740        & 1                   & 24000191  \\ 
		            & 7496370         & 1                   & 24000191  \\ 
		            & 3748185         & 24000190            & 1         \\ \hline
		23856717    & -               & -                   & 1         \\ \cline{2-4}
		            & 59970960        & 1                   & 24000191  \\ 
		            & 29985480        & 1                   & 24000191  \\ 
		            & 14992740        & 1                   & 24000191  \\ 
		            & 7496370         & 15288739            & 9199      \\ \hline
		\end{tabular}
		\end{table}
		
		De nuevo, llegamos a la misma descomposición $\displaystyle 24000191 = 9199 \cdot \frac{24000191}{9199}
		= 9199 \cdot 2609$. Una peculiaridad que debemos tener en cuenta es que dado que este algoritmo es
		probabilístico, en cada ejecución, nos saldrá una tabla distinta.
	\end{enumerate}
