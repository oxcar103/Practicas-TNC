	Consideremos $dni$ nuestro DNI módulo $65536 = 2^{16}$ y el cifrado por bloques MiniAES visto en clase:
	\begin{enumerate}
		\item Calcula $E_{dni}(\mathrm{0x01234567})$ usando el modo OFB e $IV = \mathrm{0x0001}$
		\item Calcula $E_{dni}(\mathrm{0x01234567})$ usando el modo CFB, $r = 3$ e $IV = \mathrm{0x0001}$
	\end{enumerate}
\section*{Solución}
	Para mi DNI particular se tiene que $dni = 1000\ 0001\ 0110\ 0111_2 = \mathrm{0x8167}$.
	
	Antes de pasar a la descripción de los modos de operación, repasaremos el algoritmo MiniAES visto en clase.
	
	Lo primero es notar que tiene las mismas bases matemáticas que el algoritmo AES pero tomando números y
	polinomios menores. Esto es, mientras el AES requiere 128 bits, MiniAES sólo 16 bits, y mientras que AES
	trabaja sobre $\mathbb{Z}_2(\xi)_{\xi^8+\xi^4+\xi^3+\xi+1}$, MiniAES sobre $\mathbb{Z}_2(\xi)_{\xi^4+\xi+1}$.
	
	El algoritmo básico de cifrado de MiniAES es:
	\begin{algorithm}[H]
		\begin{algorithmic}[1]
			\REQUIRE \ \\
				\texttt{$m$}, mensaje \\
				\texttt{$k$}, clave\\ \
			\STATE{\texttt{$k_0, k_1, k_2 = GenerateKeys(k)$}}
			\STATE{\texttt{$aes_0 = m$}}
			\STATE{\texttt{$aes_1 = AddRoundKey(k_0, aes_0)$}}
			\STATE{\texttt{$aes_2 = SubBytes(aes_1)$}}
			\STATE{\texttt{$aes_3 = ShiftRows(aes_2)$}}
			\STATE{\texttt{$aes_4 = MixColumns(aes_3)$}}
			\STATE{\texttt{$aes_5 = AddRoundKey(k_1, aes_4)$}}
			\STATE{\texttt{$aes_6 = SubBytes(aes_5)$}}
			\STATE{\texttt{$aes_7 = ShiftRows(aes_6)$}}
			\STATE{\texttt{$c = aes_8 = AddRoundKey(k_2, aes_7)$}}

			\PRINT{\texttt{$c$}, criptograma}
		\end{algorithmic}
		\caption{Algoritmo de cifrado MiniAES para 16 bits.}
		\label{EncMiniAES}
	\end{algorithm}
	
	Y el de descifrado sería:	
	\begin{algorithm}[H]
		\begin{algorithmic}[1]
			\REQUIRE \ \\
				\texttt{$c$}, criptograma \\
				\texttt{$k$}, clave\\ \
			\STATE{\texttt{$k_0, k_1, k_2 = GenerateKeys(k)$}}
			\STATE{\texttt{$aes_8 = c$}}
			\STATE{\texttt{$aes_7 = AddRoundKey(k_2, aes_8)$}}
			\STATE{\texttt{$aes_6 = SubBytesInv(aes_7)$}}
			\STATE{\texttt{$aes_5 = ShiftRows(aes_6)$}}
			\STATE{\texttt{$aes_4 = MixColumns(aes_5)$}}
			\STATE{\texttt{$aes_3 = AddRoundKey(MixColumns(k_1), aes_4)$}}
			\STATE{\texttt{$aes_2 = SubBytesInv(aes_3)$}}
			\STATE{\texttt{$aes_1 = ShiftRows(aes_2)$}}
			\STATE{\texttt{$m = aes_0 = AddRoundKey(k_0, aes_1)$}}

			\PRINT{\texttt{$m$}, mensaje}
		\end{algorithmic}
		\caption{Algoritmo de descifrado MiniAES para 16 bits.}
		\label{DecMiniAES}
	\end{algorithm}
	
	A continuación, desgranemos cada una de estas funciones.
	
	Empecemos por la más simple, $AddRoundKey(k_r, aes_i) = k_r \oplus aes_i$, mezcla la clave de ronda $k_r$
	con la variable temporal $aes_i$.
	
	La función $SubBytes$ se calcula como $SubBytes \left[\begin{matrix}a_0 & a_2\\ a_1 & a_3\end{matrix}\right]
	= \left[\begin{matrix}\gamma(a_0) & \gamma(a_2) \\ \gamma(a_1) & \gamma(a_3)\end{matrix}\right]$ con:
	$$\gamma(a) = a^{-1} \cdot \left(\begin{matrix} 0 & 1 & 1 & 1 \\ 1 & 1 & 1 & 0 \\ 1 & 1 & 0 & 1 \\
	1 & 0 & 1 & 1 \end{matrix}\right)_2 + \left(\begin{matrix} 0 & 0 & 1 & 1 \end{matrix}\right)_2$$
	
	La única salvedad es que por eficiencia se calculan previamente los 16 posibles valores de $\gamma(a)$ y se
	almacenan en una tabla en lugar de calcularlos en cada llamada a $SubBytes$. Además, esto nos permite también
	calcular fácilmente $SubBytesInv \left[\begin{matrix}a_0 & a_2\\ a_1 & a_3\end{matrix}\right] =
	\left[\begin{matrix}\gamma^{-1}(a_0) & \gamma^{-1}(a_2) \\ \gamma^{-1}(a_1) & \gamma^{-1}(a_3)\end{matrix}\right]$
	pues nos basta con mirar la tabla al revés.
	
	La función $ShiftRows$ sería $ShiftRows \left[\begin{matrix}a_0 & a_2\\ a_1 & a_3\end{matrix}\right]
	= \left[\begin{matrix}a_0 & a_2 \\ a_3 & a_1\end{matrix}\right]$ y es su propia inversa.
	
	La función $MixColumns \left[\begin{matrix}a_0 & a_2\\ a_1 & a_3\end{matrix}\right] =
	\left[\begin{matrix}0011 & 0010 \\ 0010 & 0011\end{matrix}\right]_2 \cdot \left[\begin{matrix}a_0 & a_2 \\
	a_3 & a_1\end{matrix}\right]$ y dado que $\left[\begin{matrix}0011 & 0010 \\ 0010 & 0011\end{matrix}\right]_2$
	es su propia inversa, $MixColumns$ también lo es.
	
	Por último, la función $GenerateKeys(k)$ sería:
	\begin{algorithm}[H]
		\begin{algorithmic}[1]
			\REQUIRE \ \\
				\texttt{$k$}, clave\\ \
			\STATE{\texttt{$k = k_0 k_1 k_2 k_3 = \omega_0 \omega_1 \omega_2 \omega_3$}}
			\STATE{\texttt{$\omega_4 = \omega_0 \oplus \gamma(\omega_3) \oplus 0001$}}
			\STATE{\texttt{$\omega_5 = \omega_1 \oplus \omega_4$}}
			\STATE{\texttt{$\omega_6 = \omega_2 \oplus \omega_5$}}
			\STATE{\texttt{$\omega_7 = \omega_3 \oplus \omega_6$}}
			\STATE{\texttt{$\omega_8 = \omega_4 \oplus \gamma(\omega_7) \oplus 0010$}}
			\STATE{\texttt{$\omega_9 = \omega_5 \oplus \omega_8$}}
			\STATE{\texttt{$\omega_{10} = \omega_6 \oplus \omega_9$}}
			\STATE{\texttt{$\omega_{11} = \omega_7 \oplus \omega_{10}$}}

			\PRINT{\texttt{$k_0 = \omega_0 \omega_1 \omega_2 \omega_3,\ k_1 = \omega_4 \omega_5 \omega_6 \omega_7,\
					k_2 = \omega_8 \omega_9 \omega_{10} \omega_{11}$}, claves de ronda}
		\end{algorithmic}
		\caption{Algoritmo de generación de claves para MiniAES.}
		\label{GenKeys}
	\end{algorithm}
	
	Dado que los modos de operación no alteran la clave, las claves de ronda permanecerán siempre iguales. En
	particular, con mi $dni$ se tiene: $k_0 = \mathrm{0x8167},\ k_1 = \mathrm{0x98E9},\ k_2 = \mathrm{0x5D3A}$.
	
	Una vez explicado MiniAES, procederemos a explicar los modos de operación tanto de forma teórica como sobre
	nuestro ejercicio particular:
	
	\begin{enumerate}
		\item El modo \textbf{Output FeedBack} abreviado como \textbf{OFB} permite imitar el comportamiento de
		un cifrado de flujo síncrono usando un cifrado en bloques mediante los siguientes algoritmos de cifrado
		y descifrado:
		\begin{algorithm}[H]
			\begin{algorithmic}[1]
				\REQUIRE \ \\
					\texttt{$k$}, clave\\
					\texttt{$m$}, mensaje\\
					\texttt{$IV$}, vector de inicialización\\ \
				\STATE{\texttt{$c = []$}}
				\STATE{\texttt{$x = IV$}}
				\FOR{\texttt{$m_i \in m.split(16\ bits)$}}
					\STATE{\texttt{$x = EncMiniAES(k, x)$}}
					\STATE{\texttt{$c += m_i \oplus x $}}
				\ENDFOR
				\PRINT{\texttt{$c.join()$}, criptograma}
			\end{algorithmic}
			\caption{Modo \textbf{OFB} de cifrado con MiniAES.}
			\label{EncOFB}
		\end{algorithm}
		
		\begin{algorithm}[H]
			\begin{algorithmic}[1]
				\REQUIRE \ \\
					\texttt{$k$}, clave\\
					\texttt{$c$}, criptograma\\
					\texttt{$IV$}, vector de inicialización\\ \
				\STATE{\texttt{$m = []$}}
				\STATE{\texttt{$x = IV$}}
				\FOR{\texttt{$c_i \in c.split(16\ bits)$}}
					\STATE{\texttt{$x = EncMiniAES(k, x)$}}
					\STATE{\texttt{$m += c_i \oplus x $}}
				\ENDFOR
				\PRINT{\texttt{$m.join()$}, mensaje}
			\end{algorithmic}
			\caption{Modo OFB de descifrado con MiniAES.}
			\label{DecOFB}
		\end{algorithm}
		
		\item El modo \textbf{Cipher FeedBack} abreviado como \textbf{CFB} permite imitar el comportamiento de
		un cifrado de flujo auto-síncrono usando un cifrado en bloques mediante los siguientes algoritmos de
		cifrado y descifrado:
		\begin{algorithm}[H]
			\begin{algorithmic}[1]
				\REQUIRE \ \\
					\texttt{$k$}, clave\\
					\texttt{$m$}, mensaje\\
					\texttt{$r$}, bits a renovar\\
					\texttt{$IV$}, vector de inicialización\\ \
				\STATE{\texttt{$c = []$}}
				\STATE{\texttt{$x = IV$}}
				\FOR{\texttt{$m_i \in m.split(r\ bits)$}}
					\STATE{\texttt{$c += c_i = m_i \oplus msb(EncMiniAES(k, x))$}}
					\STATE{\texttt{$x = lsb(x) || c_i$}}
				\ENDFOR
				\PRINT{\texttt{$c.join()$}, criptograma}
			\end{algorithmic}
			\caption{Modo \textbf{CFB} de cifrado con MiniAES.}
			\label{EncCFB}
		\end{algorithm}
		\begin{algorithm}[H]
			\begin{algorithmic}[1]
				\REQUIRE \ \\
					\texttt{$k$}, clave\\
					\texttt{$c$}, criptograma\\
					\texttt{$r$}, bits a renovar\\
					\texttt{$IV$}, vector de inicialización\\ \
				\STATE{\texttt{$m = []$}}
				\STATE{\texttt{$x = IV$}}
				\FOR{\texttt{$c_i \in c.split(r\ bits)$}}
					\STATE{\texttt{$m += c_i \oplus msb(EncMiniAES(k, x))$}}
					\STATE{\texttt{$x = lsb(x) || c_i$}}
				\ENDFOR
				\PRINT{\texttt{$m.join()$}, mensaje}
			\end{algorithmic}
			\caption{Modo \textbf{CFB} de descifrado con MiniAES.}
			\label{DecCFB}
		\end{algorithm}
	\end{enumerate}
