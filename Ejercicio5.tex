	Dado tu número $n$:
	\begin{enumerate}
		\item Sea $d$ el primer elemento de la sucesión $5, -7, 9, -11, 13,\dots$ que satisface que el símbolo
		de Jacobi $\displaystyle \left(\frac{d}{n}\right) = -1$.
		\item Con el discriminante $d$, define $r = n+1$\footnote{En esta práctica, definimos $r$ igual que en
		la práctica anterior $r = n - \displaystyle \left(\frac{d}{n}\right)$ con la diferencia de que, por la
		elección del discriminante $d$, se tiene que $\displaystyle \left(\frac{d}{n}\right) = -1$.}, $P = 1$,
		$Q = \displaystyle \frac{1-d}{4}$, el entero cuadrático $\alpha$ y las sucesiones de Lucas asociadas.
		\item Si tu $n$ fuera primo, ¿Qué debería pasarle a los términos $V_r$ y $U_r$? ¿Y a los términos
		$V_{\frac{r}{2}}$ y $U_{\frac{r}{2}}$?
		\item Calcula los términos $V_{\frac{r}{2}}$, $U_{\frac{r}{2}}$, $V_r$ y $U_r$ módulo $n$ de las sucesiones
		de Lucas con el último algoritmo iterativo.
		\item ¿Se verifica el Teorema Pequeño de Fermat (TPF) para $\alpha$? ¿Qué deduces sobre la primalidad
		de tu $n$?
	\end{enumerate}

\section*{Solución}
