%%
% Plantilla de Trabajo
% Modificación de una plantilla de Latex de Frits Wenneker para adaptarla 
% al castellano y a las necesidades de escribir informática y matemáticas.
%
% Editada por: Mario Román
%
% License:
% CC BY-NC-SA 3.0 (http://creativecommons.org/licenses/by-nc-sa/3.0/)
%%

%%%%%%%%%%%%%%%%%%%%
% Short Sectioned Assignment
% LaTeX Template
% Version 1.0 (5/5/12)
%
% This template has been downloaded from:
% http://www.LaTeXTemplates.com
%
% Original author:
% Frits Wenneker (http://www.howtotex.com)
%
% License:
% CC BY-NC-SA 3.0 (http://creativecommons.org/licenses/by-nc-sa/3.0/)
%
%%%%%%%%%%%%%%%%%%%%%

%----------------------------------------------------------------------------------------
%	PAQUETES Y CONFIGURACIÓN DEL DOCUMENTO
%----------------------------------------------------------------------------------------

%% Configuración del papel.
% fourier: Usa la fuente Adobe Utopia. (Comentando la línea usa la fuente normal)
\documentclass[paper=a4, fontsize=11pt, spanish]{scrartcl} 
\usepackage{fourier}

% Centra y formatea los títulos de sección.
% Quita la indentación de párrafos.
\usepackage{sectsty} % Allows customizing section commands
\allsectionsfont{\centering \normalfont\scshape} % Make all sections centered, the default font and small caps
\setlength\parindent{0pt} % Removes all indentation from paragraphs - comment this line for an assignment with lots of text

%% Castellano.
% noquoting: Permite uso de comillas no españolas.
% lcroman: Permite la enumeración con numerales romanos en minúscula.
% fontenc: Usa la fuente completa para que pueda copiarse correctamente del pdf.
\usepackage[spanish,es-noquoting,es-lcroman]{babel}
\usepackage[utf8]{inputenc}
\usepackage[T1]{fontenc}
\selectlanguage{spanish}

%% Matemáticas.
% Paquetes de la AMS. Para entornos de ecuaciones.
\usepackage{amsmath,amsfonts,amsthm}

% Para algoritmos
\usepackage{algorithm}
\usepackage{algorithmic}
\usepackage{amsthm}
\input{spanishAlgorithmic.tex}

% Enlaces
\usepackage[hidelinks]{hyperref}

%----------------------------------------------------------------------------------------
%	TÍTULO
%----------------------------------------------------------------------------------------
% Título con las líneas horizontales, nombres y fecha.

\newcommand{\horrule}[1]{\rule{\linewidth}{#1}} % Create horizontal rule command with 1 argument of height

\title{
  \normalfont \normalsize 
  \textsc{Universidad de Granada.} \\ [25pt] % Your university, school and/or department name(s)
  \horrule{0.5pt} \\[0.4cm] % Thin top horizontal rule
  \huge Ejercicios de Teoría de Números y Criptografía \\ % The assignment title
  \horrule{2pt} \\[0.5cm] % Thick bottom horizontal rule
}

\author{Óscar Bermúdez Garrido} % Your name

\date{\normalsize\today} % Today's date or a custom date

%----------------------------------------------------------------------------------------
%	DOCUMENTO
%----------------------------------------------------------------------------------------


\begin{document}
	\maketitle % Escribe el título
	
	\newpage
	\tableofcontents % Table de contenidos
	\newpage

	\section{Ejercicio 1}
		$d1$, $d2$, $d3$, $d4$, $d5$, $d6$, $d7$ y $d8$ serán los valores de los dígitos de tu DNI o pasaporte, donde $d1$
	es tu primer dígito. Si te faltan dígitos, añades ceros. Así, $d1d2d3d4d5d6d7d8$ será un número de 8 cifras.
	
	Dado tu número de 8 cifras, $n$:
	\begin{enumerate}
		\item Con el algoritmo de la exponencialización rápida, calcula sus $a-sucesiones$ para los 5 primeros primos.
		\item ¿Qué dice el test de Solovay-Strassen para esas 5 bases? ¿Con qué probabilidad tu $n$ es primo?
	\end{enumerate}

\section*{Solución}
	Lo primero que calcularemos es $n$, que en mi caso sería $n = 26050919 =_b 0001 1000 1101 1000 0001 0110 0111$
	
	\begin{enumerate}
		\item El algoritmo que he utilizado es el de exponencialización rápida de izquierda a derecha modificado
		para que en cada iteración $i$ calcule $a^{exp_i} \mod{n}$ en lugar de $a^{exp_i}$ para poder tratar el
		número pues de otro modo sería intratable. El algoritmo es el siguiente:
		\begin{algorithm}[H]
			\begin{algorithmic}[1]
				\REQUIRE \ \\
					\texttt{$a$}, base \\
					\texttt{$b =_b e_je_{j-1}e_{j-2}\dots e_2e_1e_0$}, exponente y su expresión en binario \\
					\texttt{$c$}, módulo\\ \
				\STATE{\texttt{$acu = 1$}}
				\STATE{\texttt{$exp = 0$}}
				\STATE{\texttt{$i = j$; $j$ es el número de dígitos en binario de $b$}}
				\WHILE{\texttt{$i > 0$}}
					\STATE{\texttt{$acu \equiv acu^2 \mod{c}$ }}
					\STATE{\texttt{$exp = 2 \cdot exp$}}
					\IF{\texttt{$e_i = 1$}}
						\STATE{\texttt{$acu \equiv acu \cdot a \mod{c}$}}
						\STATE{\texttt{$exp = exp + 1$}}
					\ENDIF
					\STATE{\texttt{$i = i - 1$}}
				\ENDWHILE
				\PRINT{\texttt{$acu$, $exp$}}
			\end{algorithmic}
			\caption{Exponencialización rápida de izquierda a derecha}
			\label{Fast-exp}
		\end{algorithm}
		
		Dado que $\displaystyle \frac{n-1}{2} = 13025459$ es impar, las $a-sucesiones$ tendrás sólo 2 términos.
		
		\begin{enumerate}
			\item Para $a = 2$ tenemos:
				$acu = 2^{13025459} \equiv 3764029$
				$acu = 2^{26050918} \equiv 17811015$
				
			\item Para $a = 3$ tenemos:
				$acu = 3^{13025459} \equiv 21679447$
				$acu = 3^{26050918} \equiv 11610658$
				
			\item Para $a = 5$ tenemos:
				$acu = 5^{13025459} \equiv 6820151$
				$acu = 5^{26050918} \equiv 22769921$
				
			\item Para $a = 7$ tenemos:
				$acu = 7^{13025459} \equiv 16034947$
				$acu = 7^{26050918} \equiv 2720332$
				
			\item Para $a = 11$ tenemos:
				$acu = 11^{13025459} \equiv 14340865$
				$acu = 11^{26050918} \equiv 22153099$
				
			Claramente, $n$ es compuesto. De hecho, no engaña a ningún $a$ primo tomado.
		\end{enumerate}
		
		\item El algoritmo que he utilizado es el de cálculo del símbolo de Jacobi\footnote{Este valor coincide
		con el símbolo de Legendre cuando $n$ es un primo impar.} excepto para el caso del primo 2. El algoritmo
		es el siguiente:
		\begin{algorithm}[H]
			\begin{algorithmic}[1]
				\REQUIRE \ \\
					\texttt{$a$}, elemento \\
					\texttt{$n$}, base \\ \
				\STATE{\texttt{$t = 1$}}
				\STATE{\texttt{$m = |n|$}}
				\STATE{\texttt{$b \equiv a \mod m$}}
				\WHILE{\texttt{$b != 0$}}
					\WHILE{\texttt{$b$ par}}
						\STATE{\texttt{$\displaystyle b = \frac{b}{2}$}}
						\IF{\texttt{$m \mod 8 \in \{3, 5\}$}}
							\STATE{\texttt{$t = -t$}}
						\ENDIF
					\ENDWHILE
					\STATE{\texttt{$(b, m) = (m, b)$}}
					\IF{\texttt{$b \equiv m \equiv 3 \mod 4$}}
						\STATE{\texttt{$b \equiv b \mod{m}$}}
						\STATE{\texttt{$t = -t$}}
					\ENDIF
				\ENDWHILE
				\IF{\texttt{$m = 1$}}
					\PRINT{\texttt{$t$}}
				\ELSE
					\PRINT{\texttt{$0$}}
				\ENDIF
			\end{algorithmic}
			\caption{Símbolo de Jacobi}
			\label{Jac-symbol}
		\end{algorithm}
	\end{enumerate} % Primer ejercicio

	\section{Ejercicio 2}
		$d1$, $d2$, $d3$, $d4$, $d5$, $d6$, $d7$ y $d8$ serán los valores de los dígitos de tu DNI o pasaporte, donde
	$d1$ es tu primer dígito distinto de cero. Si te faltan dígitos, añades ceros. Así, $d1d2d3d4d5d6d7d8$ será
	un número de 8 cifras.
	
	Dado tu número de 8 cifras, $n$:
	\begin{enumerate}
		\item Halla la factorización en primos de $n$.
		\item Sea $n_1$ el mayor de los factores primos de $n$ módulo $10^5$ y $n_2$ el segundo módulo $10^4$,
		calcula las FCS de $\sqrt{n_1}$ y $\sqrt{n_2}$.
	\end{enumerate}

\section*{Solución}
 % Segundo ejercicio

	\section{Ejercicio 3}
		Sea $n$ el número de 5 o menos cifras que has usado en el ejercicio 2 para hallar la FCS de su raíz cuadrada.
	\begin{enumerate}
		\item Aplica las ecuaciones en recurrencia para hallar el penúltimo convergente que da la solución a una
		de las ecuaciones de Pell $x^2 - n \cdot y^2 = \pm 1$.
		\item Calcula la diferencia entre tu convergente y tu raíz cuadrada. ¿Qué orden de magnitud tiene ese error?
		\item ¿Existe solución para la ecuación $x^2 - n \cdot y^2 = -1$?¿Cuál es la menor solución positiva para
		$x^2 - n \cdot y^2 = 1$?
	\end{enumerate}

\section*{Solución}
 % Tercer ejercicio

	\section{Ejercicio 4}
		Sea $n$ el siguiente número primo\footnote{Aunque conocemos este hecho, dado que la utilidad de este ejercicio
	es conocer y poner a prueba nuestro conocimiento sobre tests de primalidad, lo ignoraremos y trataremos a $n$
	como un número que queremos ver si es primo.} mayor estricto que tu DNI, así $n$ tendrá 8 cifras decimales.
	\begin{enumerate}
		\item Con $P = 3$ y $Q = -1$, define un entero cuadrático $\alpha$ y sus sucesiones de Lucas.
		\item Para el discriminante correspondiente, $\Delta = P^2 - 4Q = 13$, calcula el símbolo de Jacobi
		$\displaystyle \left(\frac{\Delta}{n}\right)$ y define $\displaystyle r = n - \left(\frac{\Delta}{n}\right)$.
		Si tu $n$ fuera primo, ¿qué debería pasarle a $\alpha^r$? ¿Y a los términos $V_r$ y $U_r$?
		\item Calcula los términos $V_r$ y $U_r$ módulo $n$ de las sucesiones de Lucas con el algoritmo de izquierda
		a derecha.
		\item ¿Se verifica el Teorema Pequeño de Fermat (TPF) para $\alpha$? ¿Qué deduces sobre la primalidad de
		tu $n$?
	\end{enumerate}

\section*{Solución}
   \begin{enumerate}
        \item Claramente, definimos $\displaystyle \alpha = \frac{P + \sqrt{\Delta}}{2}$ con $\Delta = P^2 - 4Q$,
        y tenemos que $\Delta = 3^2 - 4(-1) = 9+4 = 13$ y $\displaystyle \alpha = \frac{3 + \sqrt{13}}{2}$
        
        Podemos comprobar que nuestro $\alpha$ es un entero algebraico pues $a = \alpha + \beta, b = \alpha
        \beta \in \mathbb{Z}$ con $\displaystyle \beta = \overline{\alpha} = \frac{P - \sqrt{\Delta}}{2}$:
        $$\displaystyle a = \frac{P + \sqrt{\Delta}}{2} + \frac{P - \sqrt{\Delta}}{2} = P = 3, \qquad
        b = \frac{P + \sqrt{\Delta}}{2} \cdot \frac{P - \sqrt{\Delta}}{2} = \frac{P^2-\Delta}{2^2} =
        \frac{P^2-(P^2-4Q)}{4} = Q = -1$$
        
        De aquí deducimos que su polinomio primitivo mínimo es $f(x) = x^2 -3x -1$. A partir de aquí, podemos
        deducir que $\alpha^2 -3\alpha -1 = 0 \Leftrightarrow \alpha^2  = 3\alpha +1 \Leftrightarrow \alpha^n  =
        3\alpha^{n-1} + \alpha^{n-2}$
        
        Ahora, como $\mathbb{Q\left(\alpha\right)} = \mathbb{Q\left(\sqrt{\Delta}\right)}$ con $\left\lbrace1,
        \sqrt{\Delta}\right\rbrace$ es una base de este cuerpo, podemos escribir $\displaystyle \alpha^n =
        \frac{V_n}{2} + \frac{U_n}{2}\sqrt{\Delta}$ con $U_i, V_i \in \mathbb{Z}$.
        
        Además, podemos usar la expresión $\alpha^n  = 3\alpha^{n-1} + \alpha^{n-2}$ encontrada antes para
        concluir que $\displaystyle \frac{V_n}{2} + \frac{U_n}{2}\sqrt{\Delta} = \alpha^n  = 3\alpha^{n-1} +
        \alpha^{n-2} = 3\left(\frac{V_{n-1}}{2} + \frac{U_{n-1}}{2}\sqrt{\Delta}\right) -\left(\frac{V_{n-2}}{2} +
        \frac{U_{n-2}}{2}\sqrt{\Delta}\right) = \frac{3V_{n-1}-V_{n-2}}{2} + \frac{3U_{n-1}-U_{n-2}}{2}\sqrt{\Delta}$

        De donde concluimos que:
        $$\left\lbrace
   			V_n = 3V_{n-1}-V_{n-2} \atop
   			U_n = 3U_{n-1}-U_{n-2}
   		\right.$$
   		
   		Cumplen por tanto una relación de recurrencia, ahora sólo tenemos que calcular algunos valores para tener
   		completamente determinada la recurrencia. En nuestro caso, es fácil ver que:
		$$\left.
   			\displaystyle 1 = \alpha^0 = \frac{V_0}{2} + \frac{U_0}{2}\sqrt{\Delta} \Rightarrow V_0 = 2, \quad U_0 = 0 \atop
   			\displaystyle \frac{P + \sqrt{\Delta}}{2} = \alpha^1 = \frac{V_1}{2} + \frac{U_1}{2}\sqrt{\Delta}\Rightarrow V_1 = P = 3, \quad U_1 = 1 
   		\right.$$
   		
   		A las sucesiones de estos valores $V_n$ y $U_n$ que siguen la recurrencia que acabamos de ver es lo que
   		llamaremos \textit{Sucesiones de Lucas}.
   		
		\item Para calcular el símbolo de Jacobi $\displaystyle \left(\frac{\Delta}{n}\right)$ haremos uso del
		algorimto de cálculo explicado algunos ejercicios más atrás, por lo que no veo necesaria la explicación
		del algoritmo o del proceso pues ya se detalló en su momento, que sería el siguiente:
		$$\left(\frac{\Delta}{n}\right) =\footnote{Recordemos que para este ejercicio, nuestro $n$ era
		distinto, en lugar de ser el DNI, es el siguiente primo, en mi caso, $n = 26050967$.} \left(\frac{13}
		{26050967} \right)= \left(\frac{26050967}{13}\right) = \left(\frac{7}{13}\right) = \left(\frac{13}{7}\right)
		= \left(\frac{6}{7}\right) = \left(\frac{2}{7}\right) \cdot \left(\frac{3}{7}\right) = 1 \cdot
		\left(\frac{3}{7}\right) = -\left(\frac{7}{3}\right) = -\left(\frac{1}{3}\right) = -1$$
		
		De aquí, deducimos inmediatamente que $\displaystyle r = n - \left(\frac{\Delta}{n}\right) = 26050968$
		
		Ahora, estudiaremos los valores teóricos de $\alpha^r$, $U_r$ y $V_r$ en el caso de que $n$ fuese primo.
		Si $n$ es primo, tenemos que $\displaystyle |m| < |n| \Rightarrow \gcd(m, n) = 1 \quad \forall m \in
		\mathbb{Q}$ y podemos aplicar un resultado que se deduce del Pequeño Teorema de Fermat(PTF)\footnote{Hay
		un resultado análogo para $V_r$ y $U_r$ que básicamente dice que $U_r \equiv_n 0$ y que $V_r \equiv_n
		2\alpha^r$.}. En particular, tenemos que $|2Q\Delta| < |n|$ por lo que se cumple dicho resultado:
		$$\alpha^r = \alpha^{n-\left(\frac{\Delta}{n}\right)} \equiv_n \left\lbrace
			1 \qquad \text{Si } \left(\frac{\Delta}{n}\right) = 1 \atop
			Q \qquad \text{Si } \left(\frac{\Delta}{n}\right) = -1
		\right.$$
		
		Por lo que para nuestro caso particular se tiene que $\alpha^r \equiv_n Q = -1$. Trivialmente se deduce
		que $U_r \equiv_n 0$ y $V_r \equiv_n 2Q = -2$
		
		\item Para este apartado, reutilizaremos el algoritmo de la primera práctica con pequeñas modificaciones
		pues, aunque sirve para calcular $\alpha^r$, no es tan fácil calcular $U_r$ y $V_r$ a partir de él.
		Principalmente, tenemos que modificar la salida para que nos devuelva por separado la componente racional
		de la irracional, tener en cuenta que $V_i$ y $U_i$ son el doble de las componentes racional e irracional
		respectivamente y redefinir el producto aprovechándonos de la siguiente propiedad:
		$$\left(a+b\sqrt{e}\right)\left(c+d\sqrt{e}\right)=\left(ac+bde\right)+\left(ad+cb\right)\sqrt{e}$$
		
		Finalmente, el algoritmo quedaría:
		
		Los resultados obtenidos son:
		\begin{center}
		  \begin{tabular}{ | r | c | c |}
		    \hline
		    i	& $V_i$	& $U_i$ \\
		    \hline
			1       & 3     & 1 \\
			3       & 36    & 10 \\
			6       & 1298  & 360 \\
			12      & 1684802       & 467280 \\
			24      & 18363915      & 14055820 \\
			49      & 11672958      & 4428956 \\
			99      & 20937720      & 23911249 \\
			198     & 23382471      & 21523686 \\
			397     & 15848977      & 22985641 \\
			795     & 25678459      & 18739467 \\
			1590    & 14759824      & 15744084 \\
			3180    & 20679959      & 20945882 \\
			6360    & 6271843       & 2028203 \\
			12720   & 18332426      & 13856864 \\
			25440   & 22940247      & 25097842 \\
			50880   & 25379149      & 18394763 \\
			101761  & 463429        & 21134127 \\
			203523  & 14461565      & 17713358 \\
			407046  & 11784831      & 20336956 \\
			814092  & 3753301       & 21713753 \\
			1628185 & 5185797       & 9148345 \\
			3256371 & 11511884      & 520085 \\
			6512742 & 15819230      & 20750332 \\
			13025484        & 0     & 2635672 \\
			26050968        & 26050965      & 0 \\
		    \hline
		  \end{tabular}
		\end{center}
		
    \end{enumerate}
 % Cuarto ejercicio

	\section{Ejercicio 5}
		Dado tu número $n$:
	\begin{enumerate}
		\item Sea $d$ el primer elemento de la sucesión $5, -7, 9, -11, 13,\dots$ que satisface que el símbolo
		de Jacobi $\displaystyle \left(\frac{d}{n}\right) = -1$.
		\item Con el discriminante $d$, define $r = n+1$\footnote{En esta práctica, definimos $r$ igual que en
		la práctica anterior $r = n - \displaystyle \left(\frac{d}{n}\right)$ con la diferencia de que, por la
		elección del discriminante $d$, se tiene que $\displaystyle \left(\frac{d}{n}\right) = -1$.}, $P = 1$,
		$Q = \displaystyle \frac{1-d}{4}$, el entero cuadrático $\alpha$ y las sucesiones de Lucas asociadas.
		\item Si tu $n$ fuera primo, ¿Qué debería pasarle a los términos $V_r$ y $U_r$? ¿Y a los términos
		$V_{\frac{r}{2}}$ y $U_{\frac{r}{2}}$?
		\item Calcula los términos $V_{\frac{r}{2}}$, $U_{\frac{r}{2}}$, $V_r$ y $U_r$ módulo $n$ de las sucesiones
		de Lucas con el último algoritmo iterativo.
		\item ¿Se verifica el Teorema Pequeño de Fermat (TPF) para $\alpha$? ¿Qué deduces sobre la primalidad
		de tu $n$?
	\end{enumerate}

\section*{Solución}
 % Quinto ejercicio

	\section{Ejercicio 6}
		Sea $n$ el siguiente número primo\footnote{Aunque conocemos este hecho, dado que la utilidad de este ejercicio
	es conocer y poner a prueba nuestro conocimiento sobre tests de primalidad, lo ignoraremos y trataremos a $n$
	como un número que queremos ver si es primo.} mayor estricto que tu DNI, así $n$ tendrá 8 cifras decimales.

	\begin{enumerate}
	\item Encuentra $Q$ el primer natural mayor que 1 que satisface $d=1-4Q$ no es cuadrado perfecto, su símbolo de
	Jacobi es $\displaystyle \left(\frac{d}{n}\right) = -1$ y que además la sucesión de Lucas asociada certifica
	la primalidad de tu $n$.
	\item Encuentra el primer natural mayor que 1 que sea primitivo y, por tanto, certifique la primalidad de tu $n$.
	\end{enumerate}

\section*{Solución} % Sexto ejercicio
	
\end{document}
