	Sea $n$ el siguiente número primo mayor estricto que tu DNI$^2$, así $n$ tendrá 15 cifras decimales.
	\begin{enumerate}
		\item Repite el ejercicio anterior\footnote{Por tanto, las explicaciones quedarán relegadas pues ya se
		dieron.} para el nuevo $n$.
	\end{enumerate}
	
\section*{Solución}
	\begin{enumerate}
		\item Tenemos $n-1 = 2 \cdot 339325190372289 = 2 \cdot n_1$ y $n+1 = 2^2 \cdot 169662595186145 = 2^2 \cdot n_2$.
		
		Tras realizar las etapas del algoritmo tenemos:
		\begin{itemize}
			\item $n_1 = 3 \cdot 11 \cdot 17 \cdot 137 \cdot 4415019977$ con:
			\begin{itemize}
				\item 2 un elemento primitivo de $\mathbb{Z}^*_{3}$.
				\item 2 un elemento primitivo de $\mathbb{Z}^*_{11}$.
				\item 3 un elemento primitivo de $\mathbb{Z}^*_{17}$.
				\item 3 un elemento primitivo de $\mathbb{Z}^*_{137}$.
				\item 3 un elemento primitivo de $\mathbb{Z}^*_{4415019977}$.
				\item El algoritmo de Fermat necesita demasiados pasos, por lo que no tiene sentido utilizarlo.
				\item El algoritmo $\rho$ de Pollard pasa por:
				\begin{itemize}
					\item $n_1 = 3 \cdot 113108396790763$ en 1 paso.
					\item $n_1 = 3 \cdot 11 \cdot 10282581526433$ en 4 pasos.
					\item $n_1 = 3 \cdot 11 \cdot 17 \cdot 604857736849$ en 6 pasos.
					\item $n_1 = 3 \cdot 11 \cdot 17 \cdot 137 \cdot 4415019977$ en 17 pasos.
				\end{itemize}
			\end{itemize}
			
			\item $n_2 = 5 \cdot 19 \cdot 43 \cdot 61 \cdot 680870017$ con:
			\begin{itemize}
				\item 2 un elemento primitivo de $\mathbb{Z}^*_{5}$.
				\item 2 un elemento primitivo de $\mathbb{Z}^*_{19}$.
				\item 3 un elemento primitivo de $\mathbb{Z}^*_{43}$.
				\item 2 un elemento primitivo de $\mathbb{Z}^*_{61}$.
				\item 10 un elemento primitivo de $\mathbb{Z}^*_{680870017}$.
				\item El algoritmo de Fermat necesita demasiados pasos, por lo que no tiene sentido utilizarlo.
				\item El algoritmo $\rho$ de Pollard pasa por:
				\begin{itemize}
					\item $n_2 = 5 \cdot 33932519037229$ en 3 pasos.
					\item $n_2 = 5 \cdot 19 \cdot 1785922054591$ en 4 pasos.
					\item $n_2 = 5 \cdot 19 \cdot 43 \cdot 41533071037$ en 6 pasos.
					\item $n_2 = 5 \cdot 19 \cdot 43 \cdot 61 \cdot 680870017$ en 10 pasos.
				\end{itemize}
			\end{itemize}
		\end{itemize}
	\end{enumerate}