%%
% Plantilla de Trabajo
% Modificación de una plantilla de Latex de Frits Wenneker para adaptarla 
% al castellano y a las necesidades de escribir informática y matemáticas.
%
% Editada por: Mario Román
%
% License:
% CC BY-NC-SA 3.0 (http://creativecommons.org/licenses/by-nc-sa/3.0/)
%%

%%%%%%%%%%%%%%%%%%%%
% Short Sectioned Assignment
% LaTeX Template
% Version 1.0 (5/5/12)
%
% This template has been downloaded from:
% http://www.LaTeXTemplates.com
%
% Original author:
% Frits Wenneker (http://www.howtotex.com)
%
% License:
% CC BY-NC-SA 3.0 (http://creativecommons.org/licenses/by-nc-sa/3.0/)
%
%%%%%%%%%%%%%%%%%%%%%

%----------------------------------------------------------------------------------------
%	PAQUETES Y CONFIGURACIÓN DEL DOCUMENTO
%----------------------------------------------------------------------------------------

%% Configuración del papel.
% fourier: Usa la fuente Adobe Utopia. (Comentando la línea usa la fuente normal)
\documentclass[paper=a4, fontsize=11pt, spanish]{scrartcl} 
\usepackage{fourier}

% Centra y formatea los títulos de sección.
% Quita la indentación de párrafos.
\usepackage{sectsty} % Allows customizing section commands
\allsectionsfont{\centering \normalfont\scshape} % Make all sections centered, the default font and small caps
\setlength\parindent{0pt} % Removes all indentation from paragraphs - comment this line for an assignment with lots of text

%% Castellano.
% noquoting: Permite uso de comillas no españolas.
% lcroman: Permite la enumeración con numerales romanos en minúscula.
% fontenc: Usa la fuente completa para que pueda copiarse correctamente del pdf.
\usepackage[spanish,es-noquoting,es-lcroman]{babel}
\usepackage[utf8]{inputenc}
\usepackage[T1]{fontenc}
\selectlanguage{spanish}

%% Matemáticas.
% Paquetes de la AMS. Para entornos de ecuaciones.
\usepackage{amsmath,amsfonts,amsthm}

% Para algoritmos
\usepackage{algorithm}
\usepackage{algorithmic}
\usepackage{amsthm}
\floatname{algorithm}{Algoritmo}
\renewcommand{\listalgorithmname}{Lista de algoritmos}
\renewcommand{\algorithmicrequire}{\textbf{Entrada:}}
\renewcommand{\algorithmicensure}{\textbf{Salida:}}
\renewcommand{\algorithmicend}{\textbf{Fin}}
\renewcommand{\algorithmicif}{\textbf{Si}}
\renewcommand{\algorithmicthen}{\textbf{Entonces}}
\renewcommand{\algorithmicelse}{\textbf{En otro caso}}
\renewcommand{\algorithmicelsif}{\algorithmicelse,\ \algorithmicif}
\renewcommand{\algorithmicendif}{\algorithmicend\ \algorithmicif}
\renewcommand{\algorithmicfor}{\textbf{Para }}
\renewcommand{\algorithmicforall}{\textbf{Para cada}}
\renewcommand{\algorithmicdo}{\textbf{}}
\renewcommand{\algorithmicendfor}{\algorithmicend\ \algorithmicfor}
\renewcommand{\algorithmicwhile}{\textbf{Mientras}}
\renewcommand{\algorithmicendwhile}{\algorithmicend\ \algorithmicwhile}
\renewcommand{\algorithmicloop}{\textbf{Repetir}}
\renewcommand{\algorithmicendloop}{\algorithmicend\ \algorithmicloop}
\renewcommand{\algorithmicrepeat}{\textbf{Repetir}}
\renewcommand{\algorithmicuntil}{\textbf{Hasta que}}
\renewcommand{\algorithmicprint}{\textbf{Imprimir}} 
\renewcommand{\algorithmicreturn}{\textbf{Devolver}} 
\renewcommand{\algorithmictrue}{\textbf{Verdadero }} 
\renewcommand{\algorithmicfalse}{\textbf{Falso }} 
\renewcommand{\algorithmicand}{\textbf{Y}}
\renewcommand{\algorithmicor}{\textbf{O}}
\renewcommand{\algorithmicnot}{\textbf{No}}

% Enlaces
\usepackage[hidelinks]{hyperref}

%----------------------------------------------------------------------------------------
%	TÍTULO
%----------------------------------------------------------------------------------------
% Título con las líneas horizontales, nombres y fecha.

\newcommand{\horrule}[1]{\rule{\linewidth}{#1}} % Create horizontal rule command with 1 argument of height

\title{
  \normalfont \normalsize 
  \textsc{Universidad de Granada.} \\ [25pt] % Your university, school and/or department name(s)
  \horrule{0.5pt} \\[0.4cm] % Thin top horizontal rule
  \huge Ejercicios de Teoría de Números y Criptografía \\ % The assignment title
  \horrule{2pt} \\[0.5cm] % Thick bottom horizontal rule
}

\author{Óscar Bermúdez Garrido} % Your name

\date{\normalsize\today} % Today's date or a custom date

%----------------------------------------------------------------------------------------
%	DOCUMENTO
%----------------------------------------------------------------------------------------


\begin{document}
	\maketitle % Escribe el título
	
	\newpage
	\tableofcontents % Table de contenidos
	\newpage

	\section{Ejercicio 1}
		Consideremos $dni$ nuestro DNI módulo $65536 = 2^{16}$ y el cifrado por bloques MiniAES visto en clase:
	\begin{enumerate}
		\item Calcula $E_{dni}(\mathrm{0x01234567})$ usando el modo OFB e $IV = \mathrm{0x0001}$
		\item Calcula $E_{dni}(\mathrm{0x01234567})$ usando el modo CFB, $r = 3$ e $IV = \mathrm{0x0001}$
	\end{enumerate}
\section*{Solución}
	Para mi DNI particular se tiene que $dni = 1000\ 0001\ 0110\ 0111_2 = \mathrm{0x8167}$.
	
	Antes de pasar a la descripción de los modos de operación, repasaremos el algoritmo MiniAES visto en clase.
	
	Lo primero es notar que tiene las mismas bases matemáticas que el algoritmo AES pero tomando números y
	polinomios menores. Esto es, mientras el AES requiere 128 bits, MiniAES sólo 16 bits, y mientras que AES
	trabaja sobre $\mathbb{Z}_2(\xi)_{\xi^8+\xi^4+\xi^3+\xi+1}$, MiniAES sobre $\mathbb{Z}_2(\xi)_{\xi^4+\xi+1}$.
	
	El algoritmo básico de cifrado de MiniAES es:
	\begin{algorithm}[H]
		\begin{algorithmic}[1]
			\REQUIRE \ \\
				\texttt{$m$}, mensaje \\
				\texttt{$k$}, clave\\ \
			\STATE{\texttt{$k_0, k_1, k_2 = GenerateKeys(k)$}}
			\STATE{\texttt{$aes_0 = m$}}
			\STATE{\texttt{$aes_1 = AddRoundKey(k_0, aes_0)$}}
			\STATE{\texttt{$aes_2 = SubBytes(aes_1)$}}
			\STATE{\texttt{$aes_3 = ShiftRows(aes_2)$}}
			\STATE{\texttt{$aes_4 = MixColumns(aes_3)$}}
			\STATE{\texttt{$aes_5 = AddRoundKey(k_1, aes_4)$}}
			\STATE{\texttt{$aes_6 = SubBytes(aes_5)$}}
			\STATE{\texttt{$aes_7 = ShiftRows(aes_6)$}}
			\STATE{\texttt{$c = aes_8 = AddRoundKey(k_2, aes_7)$}}

			\PRINT{\texttt{$c$}, criptograma}
		\end{algorithmic}
		\caption{Algoritmo de cifrado MiniAES para 16 bits.}
		\label{EncMiniAES}
	\end{algorithm}
	
	Y el de descifrado sería:	
	\begin{algorithm}[H]
		\begin{algorithmic}[1]
			\REQUIRE \ \\
				\texttt{$c$}, criptograma \\
				\texttt{$k$}, clave\\ \
			\STATE{\texttt{$k_0, k_1, k_2 = GenerateKeys(k)$}}
			\STATE{\texttt{$aes_8 = c$}}
			\STATE{\texttt{$aes_7 = AddRoundKey(k_2, aes_8)$}}
			\STATE{\texttt{$aes_6 = SubBytesInv(aes_7)$}}
			\STATE{\texttt{$aes_5 = ShiftRows(aes_6)$}}
			\STATE{\texttt{$aes_4 = MixColumns(aes_5)$}}
			\STATE{\texttt{$aes_3 = AddRoundKey(MixColumns(k_1), aes_4)$}}
			\STATE{\texttt{$aes_2 = SubBytesInv(aes_3)$}}
			\STATE{\texttt{$aes_1 = ShiftRows(aes_2)$}}
			\STATE{\texttt{$m = aes_0 = AddRoundKey(k_0, aes_1)$}}

			\PRINT{\texttt{$m$}, mensaje}
		\end{algorithmic}
		\caption{Algoritmo de descifrado MiniAES para 16 bits.}
		\label{DecMiniAES}
	\end{algorithm}
	
	A continuación, desgranemos cada una de estas funciones.
	
	Empecemos por la más simple, $AddRoundKey(k_r, aes_i) = k_r \oplus aes_i$, mezcla la clave de ronda $k_r$
	con la variable temporal $aes_i$.
	
	La función $SubBytes$ se calcula como $SubBytes \left[\begin{matrix}a_0 & a_2\\ a_1 & a_3\end{matrix}\right]
	= \left[\begin{matrix}\gamma(a_0) & \gamma(a_2) \\ \gamma(a_1) & \gamma(a_3)\end{matrix}\right]$ con:
	$$\gamma(a) = a^{-1} \cdot \left(\begin{matrix} 0 & 1 & 1 & 1 \\ 1 & 1 & 1 & 0 \\ 1 & 1 & 0 & 1 \\
	1 & 0 & 1 & 1 \end{matrix}\right) + \left(\begin{matrix} 0 & 0 & 1 & 1 \end{matrix}\right)$$
	
	La única salvedad es que por eficiencia se calculan previamente los 16 posibles valores de $\gamma(a)$ y se
	almacenan en una tabla en lugar de calcularlos en cada llamada a $SubBytes$. Además, esto nos permite también
	calcular fácilmente $SubBytesInv \left[\begin{matrix}a_0 & a_2\\ a_1 & a_3\end{matrix}\right] =
	\left[\begin{matrix}\gamma^{-1}(a_0) & \gamma^{-1}(a_2) \\ \gamma^{-1}(a_1) & \gamma^{-1}(a_3)\end{matrix}\right]$
	pues nos basta con mirar la tabla al revés.
	
	La función $ShiftRows$ sería $ShiftRows \left[\begin{matrix}a_0 & a_2\\ a_1 & a_3\end{matrix}\right]
	= \left[\begin{matrix}a_0 & a_2 \\ a_3 & a_1\end{matrix}\right]$ y es su propia inversa.
	
	La función $MixColumns \left[\begin{matrix}a_0 & a_2\\ a_1 & a_3\end{matrix}\right] =
	\left[\begin{matrix}0011 & 0010 \\ 0010 & 0011\end{matrix}\right] \cdot \left[\begin{matrix}a_0 & a_2 \\
	a_3 & a_1\end{matrix}\right]$ y dado que $\left[\begin{matrix}0011 & 0010 \\ 0010 & 0011\end{matrix}\right]$
	es su propia inversa, $MixColumns$ también lo es.
	
	Por último, la función $GenerateKeys(k)$ sería:
	\begin{algorithm}[H]
		\begin{algorithmic}[1]
			\REQUIRE \ \\
				\texttt{$k$}, clave\\ \
			\STATE{\texttt{$k = k_0 k_1 k_2 k_3 = \omega_0 \omega_1 \omega_2 \omega_3$}}
			\STATE{\texttt{$\omega_4 = \omega_0 \oplus \gamma(\omega_3) \oplus 0001$}}
			\STATE{\texttt{$\omega_5 = \omega_1 \oplus \omega_4$}}
			\STATE{\texttt{$\omega_6 = \omega_2 \oplus \omega_5$}}
			\STATE{\texttt{$\omega_7 = \omega_3 \oplus \omega_6$}}
			\STATE{\texttt{$\omega_8 = \omega_4 \oplus \gamma(\omega_7) \oplus 0010$}}
			\STATE{\texttt{$\omega_9 = \omega_5 \oplus \omega_8$}}
			\STATE{\texttt{$\omega_{10} = \omega_6 \oplus \omega_9$}}
			\STATE{\texttt{$\omega_{11} = \omega_7 \oplus \omega_{10}$}}

			\PRINT{\texttt{$k_0 = \omega_0 \omega_1 \omega_2 \omega_3,\ k_1 = \omega_4 \omega_5 \omega_6 \omega_7,\
					k_2 = \omega_8 \omega_9 \omega_{10} \omega_{11}$}, claves de ronda}
		\end{algorithmic}
		\caption{Algoritmo de generación de claves para MiniAES.}
		\label{GenKeys}
	\end{algorithm}
	
	Dado que los modos de operación no alteran la clave, las claves de ronda permanecerán siempre iguales. En
	particular, con mi $dni$ se tiene: $k_0 = 0x8167,\ k_1 = 0x98E9,\ k_2 = 0x5D3A$.
	
	Una vez explicado MiniAES, procederemos a explicar los modos de operación tanto de forma teórica como sobre
	nuestro ejercicio particular:
	
	\begin{enumerate}
		\item El modo \textbf{Output FeedBack} abreviado como \textbf{OFB} permite imitar el comportamiento de
		un cifrado de flujo síncrono usando un cifrado en bloques mediante los siguientes algoritmos de cifrado
		y descifrado:
		\begin{algorithm}[H]
			\begin{algorithmic}[1]
				\REQUIRE \ \\
					\texttt{$k$}, clave\\
					\texttt{$m$}, mensaje\\
					\texttt{$IV$}, vector de inicialización\\ \
				\STATE{\texttt{$c = []$}}
				\STATE{\texttt{$x = IV$}}
				\FOR{\texttt{$m_i \in m.split(16 bits)$}}
					\STATE{\texttt{$x = EncMiniAES(k, x)$}}
					\STATE{\texttt{$c += m_i \oplus x $}}
				\ENDFOR
				\PRINT{\texttt{$c$}, criptograma}
			\end{algorithmic}
			\caption{Modo \textbf{OFB} de cifrado con MiniAES.}
			\label{EncOFB}
		\end{algorithm}
		
		\begin{algorithm}[H]
			\begin{algorithmic}[1]
				\REQUIRE \ \\
					\texttt{$k$}, clave\\
					\texttt{$c$}, criptograma\\
					\texttt{$IV$}, vector de inicialización\\ \
				\STATE{\texttt{$m = []$}}
				\STATE{\texttt{$x = IV$}}
				\FOR{\texttt{$c_i \in c.split(16 bits)$}}
					\STATE{\texttt{$x = EncMiniAES(k, x)$}}
					\STATE{\texttt{$m += c_i \oplus x $}}
				\ENDFOR
				\PRINT{\texttt{$m$}, mensaje}
			\end{algorithmic}
			\caption{Modo OFB de cifrado con MiniAES.}
			\label{DecOFB}
		\end{algorithm}
	\end{enumerate}
 % Primer ejercicio

	\section{Ejercicio 2}
		$d1$, $d2$, $d3$, $d4$, $d5$, $d6$, $d7$ y $d8$ serán los valores de los dígitos de tu DNI o pasaporte, donde
	$d1$ es tu primer dígito distinto de cero. Si te faltan dígitos, añades ceros. Así, $d1d2d3d4d5d6d7d8$ será
	un número de 8 cifras.
	
	Dado tu número de 8 cifras, $n$:
	\begin{enumerate}
		\item Halla la factorización en primos de $n$.
		\item Sea $n_1$ el mayor de los factores primos de $n$ módulo $10^5$ y $n_2$ el segundo módulo $10^4$,
		calcula las FCS de $\sqrt{n_1}$ y $\sqrt{n_2}$.
	\end{enumerate}

\section*{Solución}
	\begin{enumerate}
		\item El algoritmo que he utilizado para la descomposición en factores primos de un número dado es el
		siguiente:
		\begin{algorithm}[H]
			\begin{algorithmic}[1]
				\REQUIRE \ \\
					\texttt{$n$}, número a descomponer en primos \\ \
				\STATE{\texttt{$lim = \sqrt{n}$}}
				\STATE{\texttt{$prime = 2$}}
				\WHILE{\texttt{$prime \le lim$}}
					\WHILE{\texttt{$n \equiv 0 \mod{prime}$}}
						\PRINT{\texttt{prime}}
						\STATE{\texttt{$\displaystyle n = \frac{n}{prime}$}}
						\STATE{\texttt{$lim = \sqrt{n}$}}
					\ENDWHILE
					\STATE{\texttt{$prime = nextPrime(n)$}}
				\ENDWHILE
				
				\IF{\texttt{$n \neq 1$}}
					\PRINT{\texttt{n}}
				\ENDIF
			\end{algorithmic}
			\caption{Factorización de un número dado.}
			\label{Factors}
		\end{algorithm}
		
		En particular, para nuestro $n$ obtenemos la descomposición:
		$$n = 26050919 = 17 \cdot 19 \cdot 59 \cdot 1367$$
		
		\item Para calcular los índices de la Fracción Continua Simple(FCS) de la raíz de un número usaremos el
		siguiente algoritmo:
		\begin{algorithm}[H]
			\begin{algorithmic}[1]
				\REQUIRE \ \\
					\texttt{$n$}, número a sobre el que trabajar \\ \
				\STATE{\texttt{$q_0 = \lfloor \sqrt{n} \rfloor$}}
				\STATE{\texttt{$i = 0$}}
				\STATE{\texttt{$P_i = 0$}}
				\STATE{\texttt{$Q_i = 1$}}
				\STATE{\texttt{$q_i = q_0$}}
				\WHILE{\texttt{$q_i < 2 \cdot q_0$}}
					\STATE{\texttt{$i = i + 1$}}
					\STATE{\texttt{$P_i = q_{i-1} \cdot Q_{i-1} - P_{i-1}$}}
					\STATE{\texttt{$\displaystyle Q_i = \frac{n - P_i^2}{Q_{i-1}}$}}
					\STATE{\texttt{$\displaystyle q_i = \frac{P_i + q_0}{Q_i}$}}
					\PRINT{\texttt{$P_i, Q_i, q_i$}}
				\ENDWHILE
			\end{algorithmic}
			\caption{Algoritmo de cálculo de la FCS de la raíz de un número.}
			\label{FCS}
		\end{algorithm}
		
		Una vez mostrado el algoritmo, veamos el resultado de aplicarlo sobre $n_1 = 1367$ y $n_2 = 59$:
		\begin{itemize}
			\item Claramente, tenemos que $n_1 = 1367 = 37^2-2 = a^2 - 2$ con $a = 37$, por lo que teóricamente
			$\sqrt{n_1} = [a-1, \overline{1, a-2, 1, 2 \cdot (a-1)}] = [36, \overline{1, 35, 1, 72}]$ y efectivamente
			tenemos que:
			\begin{itemize}
				\item $\textbf{i = 0:} \quad P_0 = 0, \quad Q_0 = 1, \quad q_0 = 36$
				\item $\textbf{i = 1:} \quad P_1 = 36, \quad Q_1 = 71, \quad q_1 = 1$
				\item $\textbf{i = 2:} \quad P_2 = 35, \quad Q_2 = 2, \quad q_2 = 35$
				\item $\textbf{i = 3:} \quad P_3 = 35, \quad Q_3 = 71, \quad q_3 = 1$
				\item $\textbf{i = 4:} \quad P_4 = 36, \quad Q_4 = 1, \quad q_4 = 72$
			\end{itemize}
			
			\item Por otra parte, tenemos que $n_2 = 59$, para el cuál teóricamente no tenemos ningún patrón
			específico, sólo el común para todas las raíces que no son cuadrados perfectos, es decir, $\sqrt{n_2}
			= [q_0, \overline{\dots, 2 \cdot q_0}]$. Su descomposición sería $\sqrt{n_2} = [7, \overline{1, 2,
			7, 2, 1, 14}]$ y su proceso de cálculo sería:
			\begin{itemize}
				\item $\textbf{i = 0:} \quad P_0 = 0, \quad Q_0 = 1, \quad q_0 = 7$
				\item $\textbf{i = 1:} \quad P_1 = 7, \quad Q_1 = 10, \quad q_1 = 1$
				\item $\textbf{i = 2:} \quad P_2 = 3, \quad Q_2 = 5, \quad q_2 = 2$
				\item $\textbf{i = 3:} \quad P_3 = 7, \quad Q_3 = 2, \quad q_3 = 7$
				\item $\textbf{i = 4:} \quad P_4 = 7, \quad Q_4 = 5, \quad q_4 = 2$
				\item $\textbf{i = 5:} \quad P_5 = 3, \quad Q_5 = 10, \quad q_5 = 1$
				\item $\textbf{i = 6:} \quad P_6 = 7, \quad Q_6 = 1, \quad q_6 = 14$
			\end{itemize}
		\end{itemize}
		
	\end{enumerate}
 % Segundo ejercicio

	\section{Ejercicio 3}
		Considerando nuestro $DNI$, lo separamos en 4 cifras decimales y, si alguna de las dos partes empieza por 0,
	la rotamos a izquierda hasta que no lo sea. En el caso de que esto no sea posible porque una de esas partes
	es 0000, se puede tomar un número de 4 cifras que no empiece por 0 a elección.
	
	Sean $p$ y $q$ los primeros primos mayores o iguales que los bloques anteriores, sea $n = pq$ y $e \geq 11$
	el menor primo no divisor de $\phi(n)$\footnote{Nótese que $\phi$ es la función totiente de Euler.} y sea
	$d \equiv e^{-1} \mod \phi(n)$.
	\begin{enumerate}
		\item Cifra el mensaje $m = \mathrm{0xCAFE}$.
		\item Descifra el criptograma anterior.
		\item Intenta factorizar $n$ mediante el método $P-1$ de Pollard llegando, como máximo\footnote{Spoiler:
		Dado que el número es pequeño, no voy a respetar este límite.}, a $b = 8$.
		\item Intenta factorizar $n$ a partir de $\phi(n)$.
		\item Intenta factorizar $n$ a partir de $d$
	\end{enumerate}
\section*{Solución}
	Lo primero que calculamos son los parámetros que utilizará RSA. En particular, para mi $DNI$ se tiene que:
	\begin{itemize}
		\item $(p,\ q) = (2609,\ 9199)$
		\item $n = 24000191, \quad \phi(n) = 23988384$
		\item $e = 11, \quad d \equiv e^{-1} \mod \phi(n) = 10903811$
	\end{itemize}
	
	\begin{enumerate}
		\item Para el cifrado RSA nos basta con calcular $c = m^e \mod n$, veámoslo en nuestro caso particular:
		$$\mathrm{0xCAFE}^{11} = \mathrm{0xCAFE}^{1011_2} = \left(\left(\left(\mathrm{0xCAFE}^2\right)^2\right)
		\cdot \mathrm{0xCAFE}\right)^2 \cdot \mathrm{0xCAFE} \equiv \mathrm{0xE3CD33} \mod n$$
		
		\item Para el descifrado RSA, utilizaremos $d$ en lugar de $e$, es decir, $m = c^d \mod n$: \\
		\begin{gather*}
			\mathrm{0xE3CD33}^{10903811} = \mathrm{0xE3CD33}^{1010\ 0110\ 0110\ 0001\ 0000\ 0011_2} = \\
			\left(\left( \cdots \left(\left(\left(\mathrm{0xE3CD33}^2\right)^2 \cdot \mathrm{0xE3CD33}
			\right)^2\right)^2 \cdots \right)^2 \cdot \mathrm{0xE3CD33}\right)^2 \cdot \mathrm{0xE3CD33}
			\equiv \mathrm{0xCAFE} \mod n
		\end{gather*}
		
		Adicionalmente, y para mejorar la eficiencia del cálculo podemos calcular $m = c^{d \mod \phi(p)} \mod p$
		y $m = c^{d \mod \phi(q)} \mod q$, y resolver el sistema de congruencias resultante:
		$$\left.\begin{aligned}
		        \mathrm{0xE3CD33}^{10903811 \mod \phi(2609)} &= \mathrm{0xE3CD33}^{2371} &\equiv \mathrm{0x95B} \mod 2609\\
		        \mathrm{0xE3CD33}^{10903811 \mod \phi(9199)} &= \mathrm{0xE3CD33}^{4181} &\equiv \mathrm{0x1753} \mod 9199
		       \end{aligned}
		\right\} = \mathrm{0xCAFE} = m$$
		
		\item Antes de empezar el proceso de factorización, vamos a ver el algoritmo:
		
		\begin{algorithm}[H]
			\begin{algorithmic}[1]
				\REQUIRE \ \\
					\texttt{$n$}, base\\
					\texttt{$b_0$}, caso inicial\\ \
				\STATE{\texttt{$b = b_0$}}
				\STATE{\texttt{$a = 2^{b!} \mod n$}}
				\STATE{\texttt{$g = MCD(a-1, n)$}}
				\WHILE{\texttt{$g == 1$}}
					\STATE{\texttt{$b += 1$}}
					\STATE{\texttt{$a = a^b \mod n$}}
					\STATE{\texttt{$g = MCD(a-1, n)$}}
				\ENDWHILE
				
				\PRINT{\texttt{$(g,\ n/g)$}, factores de $n$}
			\end{algorithmic}
			\caption{Método P-1 de Pollard.}
			\label{PollardP-1}
		\end{algorithm}
		
		En particular, en mi implementación $b_0$ no es un parámetro, está fijado a 2.
		
		Vista la parte teórica, pasamos a los valores particulares de nuestro caso:
		\begin{center}
		\begin{tabular}{ | r | c | c |}
			\hline
			$b$     & $a \mod n$    & GCD \\
			\hline
			2       &  4            &  1  \\
			3       &  64           &  1  \\
			4       &  16777216     &  1  \\
			5       &  9797646      &  1  \\
			6       &  5351708      &  1  \\
			7       &  19657572     &  1  \\
			8       &  2137360      &  1  \\
			9       &  9962730      &  1  \\
			10      &  983434       &  1  \\
			\vdots  &  \vdots       &  \vdots  \\
			70      &  6326632      &  1  \\
			71      &  12973364     &  1  \\
			72      &  424119       &  1  \\
			73      &  5519401      &  9199  \\
			\hline
		\end{tabular}
		\end{center}
		
		Y de aquí deducimos que $\displaystyle 24000191 = 9199 \cdot \frac{24000191}{9199} = 9199 \cdot 2609$.
		
		\item Para explotar este dato adicional, haremos uso de resultados muy básicos:
		\begin{itemize}
			\item $n = pq$ con $p,\ q$ primos $\Rightarrow \phi(n) = (p-1) \cdot (q-1) = pq -p -q +1 = n -(p+q) +1
			\Rightarrow p+q = n+1-\phi(n)$
			\item $(x-a) \cdot (x-b) = x^2 -(a+b) \cdot x + ab$
			\item $\displaystyle a \cdot x^2 + b \cdot x + c = \frac{-b \pm \sqrt{b^2 - 4 \cdot a \cdot c}}{2 \cdot a}$
		\end{itemize}
		
		Por tanto, queda claro que si calculamos $p+q = n+1-\phi(n) = 24000191+1-23988384 = 11808$ y tomamos $2b = p+q =
		2 \cdot 5904$, tenemos que $\Delta = b^2-n = 10857025 \Rightarrow \sqrt{\Delta} = 3295$, por lo que $(p,\ q) =
		\left(b-\sqrt{\Delta},\ b+\sqrt{\Delta}\right) = (2609, 9199)$
		
		\item Si en lugar de $\phi(n)$, conocemos la clave de desencriptado $d$, podemos calcular $n = pq$ basándonos
		en la propiedad de que como $n$ no es primo, la unidad tiene raíces distintas de $1$ y $-1$:
		
		\begin{algorithm}[H]
			\begin{algorithmic}[1]
				\REQUIRE \ \\
					\texttt{$n$}, base\\
					\texttt{$e$}, clave de cifrado\\
					\texttt{$d$}, clave de descifrado\\ \
				\STATE{\texttt{$g = 1$}}
				\WHILE{\texttt{$g \in \{1, n\}$}}
					\STATE{\texttt{$k = e \cdot d -1$}}
					\STATE{\texttt{$a = rand(2, n-1)$}}
					\STATE{\texttt{$g = MCD(a, n)$}, si el elemento escogido no es coprimo, ya tenemos un factor.}
					\WHILE{\texttt{$g \in \{1, n\}$ \AND $k \equiv 0 \mod 2$}}
						\STATE{\texttt{$\displaystyle k = \frac{k}{2}$}}
						\STATE{\texttt{$a = a^k \mod n$}}
						\STATE{\texttt{$g = MCD(a-1, n)$}}
					\ENDWHILE
				\ENDWHILE
				
				\PRINT{\texttt{$(g,\ n/g)$}, factores de $n$}
			\end{algorithmic}
			\caption{Factorización de $n$ conociendo la clave $d$ de descifrado.}
			\label{AtaqueD}
		\end{algorithm}
		
	\end{enumerate}
 % Tercer ejercicio

	\section{Ejercicio 4}
		Los parámetros  de un criptosistema de ElGamal son $p = 211$ y $g = 3$, es decir, el cristosistema está
	diseñado en el cuerpo $\mathbb{F}_{211} = \mathbb{Z}_{211}$ y tomamos $g = 3$ como generador de
	$\mathbb{F}^*_{211}$. La clave pública empleada es $3^a = 109  \mod 211$. Descifra el criptograma
	$(154, \textit{dni} \mod 211)$, donde \textit{dni} es nuestro número de DNI. Para calcular los logaritmos
	discretos necesarios emplea dos de los métodos descritos en la teoría.
\section*{Solución}
	Los parámetros públicos que utilizará ElGamal son:
	\begin{itemize}
		\item $p = 211$
		\item $g = 3$
		\item $g^a = 109$
	\end{itemize}
	
	Para el cifrado ElGamal nos basta tomar el mensaje $m$ y un valor aleatorio $1 < k < p-1$ y devolver como
	criptograma $c = \left(g^k \mod p, m \cdot \left(g^a\right)^k \mod p\right)$.
	
	Para el descifrado ElGamal, se utiliza el criptograma $c = \left(x, y\right)$ y el mensaje se obtiene como
	$m = y \cdot x^{-a}$.
	
	Fácil, ¿no?
	
	Pues no, porque resulta que en nuestro ejemplo particular, actuamos como el atacante, y por tanto no conocemos
	el valor de $a$, tenemos que descubrirlo a partir de $g^a$. Para ello, veremos dos métodos ampliamente usados.
	
 % Cuarto ejercicio

	\section{Ejercicio 5}
		Dado tu número $n$:
	\begin{enumerate}
		\item Sea $d$ el primer elemento de la sucesión $5, -7, 9, -11, 13,\dots$ que satisface que el símbolo
		de Jacobi $\displaystyle \left(\frac{d}{n}\right) = -1$.
		\item Con el discriminante $d$, define $r = n+1$\footnote{En esta práctica, definimos $r$ igual que en
		la práctica anterior $r = n - \displaystyle \left(\frac{d}{n}\right)$ con la diferencia de que, por la
		elección del discriminante $d$, se tiene que $\displaystyle \left(\frac{d}{n}\right) = -1$.}, $P = 1$,
		$Q = \displaystyle \frac{1-d}{4}$, el entero cuadrático $\alpha$ y las sucesiones de Lucas asociadas.
		\item Si tu $n$ fuera primo, ¿Qué debería pasarle a los términos $V_r$ y $U_r$? ¿Y a los términos
		$V_{\frac{r}{2}}$ y $U_{\frac{r}{2}}$?
		\item Calcula los términos $V_{\frac{r}{2}}$, $U_{\frac{r}{2}}$, $V_r$ y $U_r$ módulo $n$ de las sucesiones
		de Lucas con el último algoritmo iterativo.
		\item ¿Se verifica el Teorema Pequeño de Fermat (TPF) para $\alpha$? ¿Qué deduces sobre la primalidad
		de tu $n$?
	\end{enumerate}

\section*{Solución}
	\begin{enumerate}
		\item Para calcular el símbolo de Jacobi $\displaystyle \left(\frac{d}{n}\right)$\footnote{Recordemos que
		para este ejercicio, nuestro $n$ era distinto, en lugar de ser el DNI, es $n = 678650380744579$.} haremos
		uso del algorimto \ref{Jac-symbol} para cada uno de los valores hasta encontrar el que nos devuelva el
		valor deseado:
		$$\left(\frac{5}{n}\right) = \left(\frac{5}{678650380744579} \right)= \left(\frac{678650380744579}{5}\right)
		= \left(\frac{4}{5}\right) = \left(\frac{2^2}{5}\right) = \left(\frac{2}{5}\right)^2 = 1$$
		
		$$\displaystyle \left(\frac{-7}{n}\right) = \left(\frac{678650380744572}{678650380744579} \right)=
		-\left(\frac{339325190372286}{678650380744579} \right) = \left(\frac{169662595186143}{678650380744579} \right)
		= \atop \displaystyle -\left(\frac{678650380744579}{169662595186143} \right) = -\left(\frac{7}
		{169662595186143} \right) = \left(\frac{169662595186143}{7} \right) = \left(\frac{3}{7} \right) =
		-\left(\frac{7}{3} \right) = -\left(\frac{1}{3} \right) = -1$$
		
		\item Dado que $d = -7$, primero definimos $r = n+1 = 678650380744580$.
		
		Claramente, definimos $\displaystyle \alpha = \frac{P + \sqrt{\Delta}}{2}$ con $\Delta = P^2 - 4Q$,
		y tenemos que $\displaystyle \Delta = 1^2 - 4\left(\frac{1-d}{4}\right) = 1-8 = -7$ y $\displaystyle
		\alpha = \frac{1 + \sqrt{-7}}{2}$
		
		Podemos comprobar que nuestro $\alpha$ es un entero algebraico pues $a = \alpha + \beta, b = \alpha
		\beta \in \mathbb{Z}$ con $\displaystyle \beta = \overline{\alpha} = \frac{P - \sqrt{\Delta}}{2}$:
		$$\displaystyle a = \frac{P + \sqrt{\Delta}}{2} + \frac{P - \sqrt{\Delta}}{2} = P = 1, \qquad
		b = \frac{P + \sqrt{\Delta}}{2} \cdot \frac{P - \sqrt{\Delta}}{2} = \frac{P^2-\Delta}{2^2} =
		\frac{P^2-(P^2-4Q)}{4} = Q = \left(\frac{1-d}{4}\right) = 2$$
		
		De aquí deducimos que su polinomio primitivo mínimo es $f(x) = x^2 -x +2$. A partir de aquí, podemos
		deducir que $\alpha^2 -\alpha +2 = 0 \Leftrightarrow \alpha^2  = \alpha -2 \Leftrightarrow \alpha^n =
		\alpha^{n-1} - 2\alpha^{n-2}$
		
		Ahora, como $\mathbb{Q\left(\alpha\right)} = \mathbb{Q\left(\sqrt{\Delta}\right)}$ con $\left\lbrace1,
		\sqrt{\Delta}\right\rbrace$ es una base de este cuerpo, podemos escribir $\displaystyle \alpha^n =
		\frac{V_n}{2} + \frac{U_n}{2}\sqrt{\Delta}$ con $U_i, V_i \in \mathbb{Z}$.
		
		Además, podemos usar la expresión $\alpha^n  = \alpha^{n-1} - 2\alpha^{n-2}$ encontrada antes para
		concluir que $\displaystyle \frac{V_n}{2} + \frac{U_n}{2}\sqrt{\Delta} = \alpha^n  = \alpha^{n-1} -
		2\alpha^{n-2} = \left(\frac{V_{n-1}}{2} + \frac{U_{n-1}}{2}\sqrt{\Delta}\right) -2\left(\frac{V_{n-2}}{2} +
		\frac{U_{n-2}}{2}\sqrt{\Delta}\right) = \frac{V_{n-1}-2V_{n-2}}{2} + \frac{U_{n-1}-2U_{n-2}}{2}\sqrt{\Delta}$
		
		De donde concluimos que:
		$$\left\lbrace
			V_n = V_{n-1}-2V_{n-2} \atop
			U_n = U_{n-1}-2U_{n-2}
		\right.$$
		
		Cumplen por tanto una relación de recurrencia, ahora sólo tenemos que calcular algunos valores para tener
		completamente determinada la recurrencia. En nuestro caso, es fácil ver que:
		$$\left.
			\displaystyle 1 = \alpha^0 = \frac{V_0}{2} + \frac{U_0}{2}\sqrt{\Delta} \Rightarrow V_0 = 2, \quad U_0 = 0 \atop
			\displaystyle \frac{P + \sqrt{\Delta}}{2} = \alpha^1 = \frac{V_1}{2} + \frac{U_1}{2}\sqrt{\Delta}\Rightarrow V_1 = P = 1, \quad U_1 = 1 
		\right.$$
		
		A las sucesiones de estos valores $V_n$ y $U_n$ que siguen la recurrencia que acabamos de ver es lo que
		llamaremos \textit{Sucesiones de Lucas}.
		
		\item Si $n$ es primo, tenemos que $\displaystyle |m| < |n| \Rightarrow \gcd(m, n) = 1 \quad \forall m \in
		\mathbb{Q}$ y podemos aplicar un resultado que se deduce del Pequeño Teorema de Fermat(PTF)\footnote{Hay un
		resultado análogo para $V_r$ y $U_r$ que básicamente dice que $U_r \equiv_n 0$ y que $V_r \equiv_n 2\alpha^r$.}.
		En particular, tenemos que $|2Q\Delta| < |n|$ por lo que se cumple dicho resultado:
		$$\alpha^r = \alpha^{n-\left(\frac{\Delta}{n}\right)} \equiv_n \left\lbrace
			1 \qquad \text{Si } \left(\frac{\Delta}{n}\right) = 1 \atop
			Q \qquad \text{Si } \left(\frac{\Delta}{n}\right) = -1
		\right.$$
		
		Por lo que para nuestro caso particular se tiene que $\displaystyle \alpha^r \equiv_n Q = \left(\frac{1-d}
		{4}\right) = 2$. Trivialmente se deduce que $U_r \equiv_n 0$ y $V_r \equiv_n 2Q = 4$
		
		
		\item Para calcular los valores $V_i$ y $U_i$ de la sucesión de Lucas(FCS) de $\alpha$ usaremos el
		siguiente algoritmo:
		\begin{algorithm}[H]
			\begin{algorithmic}[1]
				\REQUIRE \ \\
					\texttt{($P$,$Q$)}, parámetros de definición de $\alpha$ \\
					\texttt{$b =_b e_je_{j-1}e_{j-2}\dots e_2e_1e_0$}, exponente y su expresión en binario \\
					\texttt{$c$}, módulo\\ \
				\STATE{\texttt{$k = 0$}}
				\STATE{\texttt{$U_k = 0$}}
				\STATE{\texttt{$U_{k+1} = 1$}}
				\STATE{\texttt{$i = j$; $j$ es el número de dígitos en binario de $b$}}
				\WHILE{\texttt{$i > 0$}}
					\IF{\texttt{$e_i = 0$}}
						\STATE{\texttt{$(U_k, U_{k+1})= \left(2 U_k U_{k+1} - P U_k^2, \quad U_{k+1}^2 - Q U_k^2 \right) \mod c$}}
						\STATE{\texttt{$k = 2 \cdot k$}}
					\ELSE
						\STATE{\texttt{$(U_k, U_{k+1}) = \left(U_{k+1}^2 - Q U_k^2, \quad P U_{k+1}^2 - 2 Q U_k U_{k+1}\right) \mod c$}}
						\STATE{\texttt{$k = 2\cdot k + 1$}}
					\ENDIF
					\STATE{\texttt{$i = i - 1$}}
				\ENDWHILE
				\STATE{\texttt{$V_k = 2 U_{k+1} - P U_k \mod c$}}
				\PRINT{\texttt{$V_k$, $U_k$}}
			\end{algorithmic}
			\caption{Algoritmo de cálculo de la Sucesión de Lucas.}
			\label{Lucas-Suc}
		\end{algorithm}
		
		Los resultados obtenidos son:
		
		\begin{center}
			\begin{tabular}{ | r | c | c |}
				\hline
				i               & $V_i$           & $U_i$ \\
				\hline
				1               & 1               & 1 \\
				2               & 678650380744576 & 1 \\
				4               & 1               & 678650380744576 \\
				9               & 678650380744574 & 678650380744562 \\
				19              & 678650380743782 & 678650380744122 \\
				38              & 678650380331212 & 364229 \\
				77              & 337519103051    & 678385658851458 \\
				154             & 109375826287404 & 602565117149484 \\
				308             & 314372395995960 & 174542774014619 \\
				617             & 58039114661663  & 429192061200943 \\
				1234            & 107928710961106 & 551777949295122 \\
				2468            & 173384247859610 & 212210833522312 \\
				4937            & 366727320354188 & 529622479461063 \\
				9875            & 505676002010661 & 35385039382450 \\
				19751           & 102607950624073 & 589363723379602 \\
				39502           & 643340590173009 & 145092936375106 \\
				79005           & 170746732907050 & 61261644998366 \\
				158010          & 118401632948461 & 429157243036396 \\
				316021          & 397030484411688 & 570950288579416 \\
				632042          & 300481686194155 & 429410014482951 \\
				1264084         & 204971110919891 & 565762489224266 \\
				2528169         & 205528253033002 & 11298317314980 \\
				5056339         & 653791844282549 & 99791410308675 \\
				10112678        & 248067159569544 & 548776079373177 \\
				20225357        & 293048956025566 & 495082765577251 \\
				40450714        & 505417363631563 & 21391568038419 \\
				80901429        & 244273233772495 & 581989680265375 \\
				161802859       & 616720355492241 & 614229905475822 \\
				323605718       & 677684833034733 & 3681728904872 \\
				647211437       & 384717007472382 & 440081364188387 \\
				1294422875      & 311556458941962 & 468284712345335 \\
				2588845751      & 399519376303312 & 360833625919937 \\
				5177691503      & 128552669325795 & 248444453197737 \\
				10355383006     & 572489874121131 & 8385172921419 \\
				20710766013     & 289593388299130 & 371541400459837 \\
				41421532027     & 648966171804920 & 292401549856432 \\
				82843064055     & 344933393899221 & 411709447822638 \\
				165686128111    & 242032130439575 & 324117416248409 \\
				331372256222    & 548245442239833 & 652647234407392 \\
				662744512445    & 633132408951868 & 503904479244275 \\
				1325489024891   & 566661497060214 & 666226619950587 \\
				2650978049783   & 564118422945220 & 442443178508846 \\
				5301956099567   & 519463458259509 & 400839558748825 \\
				10603912199134  & 527143422327055 & 587831159199299 \\
				21207824398268  & 55227440411499  & 163953041576294 \\
				42415648796536  & 447210282631221 & 554866031163259 \\
				84831297593072  & 439417942925372 & 619271223149976 \\
				169662595186145 & 37043558831447  & 319267259707302 \\
				339325190372290 & 0               & 244551627273197 \\
				678650380744580 & 4               & 0 \\
				\hline
			\end{tabular}
		\end{center}
	\end{enumerate}
 % Quinto ejercicio

	\section{Ejercicio 6}
		Sea $n$ el siguiente número primo\footnote{Aunque conocemos este hecho, dado que la utilidad de este ejercicio
	es conocer y poner a prueba nuestro conocimiento sobre tests de primalidad, lo ignoraremos y trataremos a $n$
	como un número que queremos ver si es primo.} mayor estricto que tu DNI, así $n$ tendrá 8 cifras decimales.

	\begin{enumerate}
		\item Encuentra $Q$ el primer natural mayor que 1 que satisface $d=1-4Q$ no es cuadrado perfecto, su símbolo de
		Jacobi es $\displaystyle \left(\frac{d}{n}\right) = -1$ y que además la sucesión de Lucas asociada certifica
		la primalidad de tu $n$.
		\item Encuentra el primer natural mayor que 1 que sea primitivo y, por tanto, certifique la primalidad de tu $n$.
	\end{enumerate}

\section*{Solución}
	\begin{enumerate}
		\item En este apartado, usaremos los conocimientos que ya tenemos del cálculo del símbolo de Jacobi (algoritmo \ref{Jac-symbol}) y del cálculo rápido de términos de la sucesión de Lucas (algoritmo \ref{Lucas-Suc}) en el siguiente algoritmo:
		
		\begin{algorithm}[H]
		\begin{algorithmic}[1]
			\REQUIRE \ \\
				\texttt{$P$}, parámetro de definición de $\alpha$ \\
				\texttt{$n$}, número al que hacer el test \\
			\STATE{\texttt{$Q = 2$, valor inicial de la búsqueda}}
			\STATE{\texttt{$Encontrado = \FALSE$}}
			\WHILE{\texttt{$Encontrado == \FALSE$ \AND $Q < r$}}
				\STATE{\texttt{$Encontrado = \TRUE$}}
				\STATE{\texttt{$d = P^2-4Q$}}
				\STATE{\texttt{$s = Jac\_symbol(d, n)$}}
				\IF{\texttt{$s == 1$}}
					\PRINT{\texttt{$Q$, $s$}}
					\STATE{\texttt{$Encontrado = \FALSE$}}
				\ELSE
					\FOR{\texttt{$p_i \in Prime\_Factors(n+1)$}}
						\IF{\texttt{$U_\frac{n+1}{p_i} \equiv_n 0$}}
							\PRINT{\texttt{$Q$, $U_\frac{n+1}{p_i} \equiv_n 0$}}
							\STATE{\texttt{$Encontrado = \FALSE$}}
						\ENDIF
					\ENDFOR
				\ENDIF
				\IF{\texttt{$Encontrado == \TRUE$}}
					\PRINT{\texttt{$Q$, $d$, $s$}}
					\FOR{\texttt{$p_i \in Prime_Factors(n+1)$}}
						\PRINT{\texttt{$U_\frac{n+1}{p_i} \mod n$}}
					\ENDFOR
				\ELSE
					\STATE{\texttt{$Q += 1$}}
				\ENDIF
			\ENDWHILE
			\PRINT{\texttt{Aceptado $Q$}}
		\end{algorithmic}
		\caption{Test de primalidad usando sucesiones de Lucas.}
		\label{Primarity}
		\end{algorithm}
			
	\end{enumerate}
 % Sexto ejercicio

	\section{Ejercicio 7}
		Sea $n$ el siguiente número primo mayor estricto que tu DNI, así $n$ tendrá 8 cifras decimales.

	\begin{enumerate}
		\item Extrae los factores potencias de 2 de $n-1$ y de $n+1$. Toma $n_1$ y $n_2$ los cofactores impares
		respectivos.
		\item Pasa el test de Miller-Rabin para varias bases y si fuera necesario el de Solovay-Strassen para ver
		si $n_1$ y $n_2$ son compuestos.
		\begin{enumerate}
			\item En caso de que hayas encontrado un certificado de composición para $n_1$ y $n_2$, aplica el método
			$\rho$ de Pollard o el de Fermat para encontrar factores de ambos.
			\item Si los factores son grandes y no has demostrado que sean compuestos, encuentra un elemento primitivo
			o una sucesión de Lucas para certificar su primalidad.
		\end{enumerate}
		\item Aplica recursivamente lo anterior hasta encontrar las factorizaciones en primos de $n-1$ y $n+1$.
	\end{enumerate}

\section*{Solución}
	\begin{enumerate}
		\item Lo primero es ver las potencias de nuestros $n-1$ y $n+1$, esto básicamente consisten en dividir
		por 2 mientras que sea par, no tiene mucha magia.
		
		Obtenemos que $n-1 = 26050966 = 2 \cdot 13025483 = 2 \cdot n_1$ y que $n+1 = 26050968 = 2^3 \cdot 3256371
		= 2^3 \cdot n_2$. Tenemos así que $n_1 = 13025483$ y que $n_2 = 3256371$.
		
		\item Para realizar el test de Miller-Rabin, es necesario comprobar si se cumple que las únicas raíces de
		la unidad en $\mathbb{Z}_n$ son 1 y -1 puesto que si $n$ es primo, $\mathbb{Z}^*_n$ es un cuerpo cíclico.
		Por tanto, como por el Teorema Pequeño de Fermat (TPF) tenemos que para cualquier elemento $a \in
		\mathbb{Z}^*_n$ se tiene que $a^{n-1} \equiv_n 1$ y sabemos que $n-1$ es par, tenemos una serie de raíces
		fáciles de calcular pues nos basta con ir dividiendo entre 2 $n-1$ hasta llegar a un $m$ impar. En cada
		paso, comprobaremos que las raíces de la unidad son 1 ó -1, si nos saliera otro valor, estaríamos seguros
		de que $n$ no es primo. El algoritmo es el siguiente\footnote{Este algoritmo es perfectamente funcional
		pero terrible en eficiencia, por lo que en la práctica se suele implementar al revés, es decir, calculando
		primero el $m$ impar y avanzando hasta $n-1$.}:
		
		\begin{algorithm}[H]
		\begin{algorithmic}[1]
			\REQUIRE \ \\
				\texttt{$n$}, número al que hacer el test \\
				\texttt{$p$}, primo con el que comprobar \\ \
			\IF{\texttt{$p^{n-1} \not\equiv_n 1$}}
				\PRINT{\texttt{No es primo}}
			\ELSE
				\STATE{\texttt{$m = \displaystyle \frac{n-1}{2}$}}
				\STATE{\texttt{$Certeza = \FALSE$}}
				\WHILE{\texttt{$m$ par \AND \NOT $Certeza$}}
					\IF{\texttt{$p^m \equiv_n -1$}}
						\PRINT{\texttt{Es primo}}
						\STATE{\texttt{$Certeza = \TRUE$}}
					\ELSIF{\texttt{$p^m \not\equiv_n 1$}}
						\PRINT{\texttt{No es primo}}
						\STATE{\texttt{$Certeza = \TRUE$}}
					\ENDIF
					\STATE{\texttt{$m = \displaystyle \frac{m}{2}$}}
				\ENDWHILE
				\IF{\texttt{$Certeza == \FALSE$}}
					\IF{\texttt{$p^m \in \{-1, 1\} \mod{n}$}}
						\PRINT{\texttt{Es primo}}
					\ELSE
						\PRINT{\texttt{No es primo}}
					\ENDIF
				\ENDIF
			\ENDIF
		\end{algorithmic}
		\caption{Test de Miller-Rabin.}
		\label{Miller-Rabin}
		\end{algorithm}
		
		Claramente, ni $n_1$ ni $n_2$ son compuestos pues fallan en la primera prueba\footnote{Nótese que deberían
		hacerse varias pruebas antes de declarar un número posible primo pero basta hacer una que nos diga que es
		compuesto para poder parar y tener seguridad de que es compuesto, como ha sido nuestro caso en el primer
		paso}:
		\begin{itemize}
			\item $2^{n_1-1} \equiv_{n_1} 9704949 \neq 1$
			\item $2^{n_2-1} \equiv_{n_2} 2931403 \neq 1$
		\end{itemize}
		
		Llegados a este paso, quedarían dos vertientes en función de la salida del test de Miller-Rabin: compuesto
		o posible primo:
		
		\begin{enumerate}
			\item En caso de que sea compuesto, aplicaremos el método de Fermat o el $\rho$ de Pollard para
			encontrar factores de ambos.
			
			Empezaremos viendo el método de Fermat, cuyo algoritmo sería:
			\begin{algorithm}[H]
			\begin{algorithmic}[1]
				\REQUIRE \ \\
					\texttt{$n$}, número a descomponer \\ \
				\STATE{\texttt{$s = \lfloor \sqrt{n} \rfloor$}}
				\IF{\texttt{$s^2 - n == 0$}}
					\PRINT{\texttt{$\{s, s\}$}}
				\ELSE
					\STATE{\texttt{$s = s+1$}}
					\STATE{\texttt{$r = s^2 - n$}}
					\STATE{\texttt{$u = 2s + 1$}}
					\STATE{\texttt{$v = 1$}}
					\WHILE{\texttt{$r \neq 0$}}
						\IF{\texttt{$r > 0$}}
							\STATE{\texttt{$r = r - v$}}
							\STATE{\texttt{$v = v + 2$}}
						\ELSIF{\texttt{$r < 0$}}
							\STATE{\texttt{$r = r + u$}}
							\STATE{\texttt{$u = u + 2$}}
						\ENDIF
					\ENDWHILE
					\PRINT{\texttt{$\displaystyle n = \frac{u-v}{2} \cdot \frac{u+v-2}{2}$}}
				\ENDIF
			\end{algorithmic}
			\caption{Método de factorización de Fermat.}
			\label{Fermat-factors}
			\end{algorithm}
			
			El problema de este algoritmo salta a la vista pues necesitamos 122851 pasos para descomponer
			$n_1 = 103 \cdot 126461$ y 194 pasos para descomponer $n_2 = 1629 \cdot 1999$. Por lo que en la
			práctica no se suele usar.
			
			Sin embargo, el algoritmo $\rho$ de Pollard suele tener mejores resultados. Su definición sería la
			siguiente \footnote{Nótese que hemos usado la función $f_m: \mathbb{N} \rightarrow \mathbb{N}$
			con $z \mapsto z^2+1 \mod{m}$ aunque la función original era la dada por $z \mapsto z^2-1 \mod{m}$}:
			\begin{algorithm}[H]
			\begin{algorithmic}[1]
				\REQUIRE \ \\
					\texttt{$n$}, número a descomponer \\ \
				\STATE{\texttt{$x = 1, \quad y = 1, \quad d = 1$, valores iniciales}}
				\WHILE{\texttt{$d == 1$}}
					\STATE{\texttt{$x = f(x)$}}
					\STATE{\texttt{$y = f(f(y))$}}
					\STATE{\texttt{$d = GCD(|x-y|, n)$}}
				\ENDWHILE
				\IF{\texttt{$d == n$}}
					\PRINT{\texttt{$n$ es primo}}
				\ELSE
					\PRINT{\texttt{$\displaystyle n = d \cdot \frac{n}{d}$}}
				\ENDIF
			\end{algorithmic}
			\caption{Método de factorización $\rho$ de Pollard.}
			\label{Rho-Pollard}
			\end{algorithm}
			
			Podemos descomponer $n_1 = 103 \cdot 126461$ mediante:
			\begin{center}
			\begin{tabular}{ | r | c | c | c |}
				\hline
				i   & $x$       & $y$       & GCD \\
				\hline
				0   & 1         & 1         & 1 \\
				1   & 2         & 5         & 1 \\
				2   & 5         & 677       & 1 \\
				3   & 26        & 4424560   & 1 \\
				4   & 677       & 476428    & 1 \\
				5   & 458330    & 3436304   & 1 \\
				6   & 4424560   & 7695078   & 1 \\
				7   & 3365853   & 11721924  & 1 \\
				8   & 476428    & 9496733   & 1 \\
				9   & 1572427   & 12797088  & 1 \\
				10  & 3436304   & 9648514   & 1 \\
				11  & 11719665  & 3898809   & 1 \\
				12  & 7695078   & 11911865  & 1 \\
				13  & 1969078   & 7882229   & 1 \\
				14  & 11721924  & 5631328   & 103 \\
				\hline
			\end{tabular}
			\end{center}

			Para descomponer $n_2 = 3 \cdot 1085457$ simplemente tenemos:
			\begin{center}
			\begin{tabular}{ | r | c | c | c |}
				\hline
				i   & $x$       & $y$       & GCD \\
				\hline
				0   & 1         & 1         & 1 \\
				1   & 2         & 5         & 3 \\
				\hline
			\end{tabular}
			\end{center}

			\item En caso de que sea un posible primo, se probaría mediante los algoritmos \ref{Primarity} usando
			sucesiones de Lucas u \ref{Primitive} de búsqueda de elementos primitivos.
			
			Pero en este paso, ambos nos salían compuestos.
		\end{enumerate}
		
		\item Finalmente, tras repetir los pasos anteriores tenemos:
		\begin{itemize}
			\item $n_1 = 103 \cdot 126461$ con:
			\begin{itemize}
				\item 5 un elemento primitivo de $\mathbb{Z}^*_{103}$
				\item 2 un elemento primitivo de $\mathbb{Z}^*_{126461}$
			\end{itemize}
			Al tener sólo dos componentes, ambos algoritmos pasan por los mismos números.
			
			\item $n_2 = 3^2 \cdot 181 \cdot 1999$ con:
			\begin{itemize}
				\item 2 un elemento primitivo de $\mathbb{Z}^*_{3}$
				\item 2 un elemento primitivo de $\mathbb{Z}^*_{181}$
				\item 3 un elemento primitivo de $\mathbb{Z}^*_{1999}$
				\item El algoritmo de Fermat pasa por:
				\begin{itemize}
					\item $n_2 = 1629 \cdot 1999$ en 194 pasos.
					\item $n_2 = 9 \cdot 181 \cdot 1999$ en 140 pasos.
					\item $n_2 = 3^2 \cdot 181 \cdot 1999$ en 1 paso.
				\end{itemize}
				
				\item El algoritmo $\rho$ de Pollard pasa por:
				\begin{itemize}
					\item $n_2 = 3 \cdot 1085457$ en 1 paso.
					\item $n_2 = 3^2 \cdot 361819$ en 1 paso.
					\item $n_2 = 3^2 \cdot 181 \cdot 1999$ en 14 pasos.
				\end{itemize}
			\end{itemize}
			
		\end{itemize}
	\end{enumerate} % Séptimo ejercicio

	\section{Ejercicio 8}
		Sea $n$ el siguiente número primo mayor estricto que tu DNI$^2$, así $n$ tendrá 15 cifras decimales.
	\begin{enumerate}
		\item Repite el ejercicio anterior\footnote{Por tanto, las explicaciones quedarán relegadas pues ya se
		dieron.} para el nuevo $n$.
	\end{enumerate}
	
\section*{Solución}
	\begin{enumerate}
		\item Tenemos $n-1 = 2 \cdot 339325190372289 = 2 \cdot n_1$ y $n+1 = 2^2 \cdot 169662595186145 = 2^2 \cdot n_2$.
		
		Tras realizar las etapas del algoritmo tenemos:
		\begin{itemize}
			\item $n_1 = 3 \cdot 11 \cdot 17 \cdot 137 \cdot 4415019977$ con:
			\begin{itemize}
				\item 2 un elemento primitivo de $\mathbb{Z}^*_{3}$.
				\item 2 un elemento primitivo de $\mathbb{Z}^*_{11}$.
				\item 3 un elemento primitivo de $\mathbb{Z}^*_{17}$.
				\item 3 un elemento primitivo de $\mathbb{Z}^*_{137}$.
				\item 3 un elemento primitivo de $\mathbb{Z}^*_{4415019977}$.
				\item El algoritmo de Fermat necesita demasiados pasos, por lo que no tiene sentido utilizarlo.
				\item El algoritmo $\rho$ de Pollard pasa por:
				\begin{itemize}
					\item $n_1 = 3 \cdot 113108396790763$ en 1 paso.
					\item $n_1 = 3 \cdot 11 \cdot 10282581526433$ en 4 pasos.
					\item $n_1 = 3 \cdot 11 \cdot 17 \cdot 604857736849$ en 6 pasos.
					\item $n_1 = 3 \cdot 11 \cdot 17 \cdot 137 \cdot 4415019977$ en 17 pasos.
				\end{itemize}
			\end{itemize}
			
			\item $n_2 = 5 \cdot 19 \cdot 43 \cdot 61 \cdot 680870017$ con:
			\begin{itemize}
				\item 2 un elemento primitivo de $\mathbb{Z}^*_{5}$.
				\item 2 un elemento primitivo de $\mathbb{Z}^*_{19}$.
				\item 3 un elemento primitivo de $\mathbb{Z}^*_{43}$.
				\item 2 un elemento primitivo de $\mathbb{Z}^*_{61}$.
				\item 10 un elemento primitivo de $\mathbb{Z}^*_{680870017}$.
				\item El algoritmo de Fermat necesita demasiados pasos, por lo que no tiene sentido utilizarlo.
				\item El algoritmo $\rho$ de Pollard pasa por:
				\begin{itemize}
					\item $n_2 = 5 \cdot 33932519037229$ en 3 pasos.
					\item $n_2 = 5 \cdot 19 \cdot 1785922054591$ en 4 pasos.
					\item $n_2 = 5 \cdot 19 \cdot 43 \cdot 41533071037$ en 6 pasos.
					\item $n_2 = 5 \cdot 19 \cdot 43 \cdot 61 \cdot 680870017$ en 10 pasos.
				\end{itemize}
			\end{itemize}
		\end{itemize}
	\end{enumerate} % Octavo ejercicio

	\section{Ejercicio 9}
		Sea $p$ el siguiente número primo mayor estricto que tu DNI y sea la curva elíptica $E: y^2 = x^3 + 4x$.
	\begin{enumerate}
		\item Calcula el número de puntos módulo $p$, que denotaremos $M_p(E)$.
		\item Dado el punto $P = (2,4)$, calcula el orden en el grupo abeliano de la curva elíptica $E$.
	\end{enumerate}
	
\section*{Solución}
	\begin{enumerate}
		\item Para calcular el número de puntos módulos $p$ podríamos basarnos en un resultado visto en teoría
		en el cuál si $p$ es un primo impar, se tiene una expresión para calcularlo:
		$$\displaystyle M_p(E) = p + 1 + \sum_{x=0}^{p-1} \left( \frac{x^3 + ax^2 + bx + c}{p} \right)$$
		
		Sin embargo, este resultado, aunque eficaz, no es eficiente dado el gran número de veces que es necesario
		aplicar el símbolo de Jacobi.
		
		Por ello, nos aprovecharemos de otro resultado que tiene como condiciones:
		\begin{itemize}
			\item $q \equiv_4 3$ primo.
			\item La curva elíptica $F$ sea de la forma $y^2 = x^3 + dx$
			\item $d \in \mathbb{Z}$ no múltiplo de $q$
		\end{itemize}
		Entonces, se tiene que $M_q(F) = q+1$.

		Claramente, nuestro $p$, nuestra $E$ y nuestro $b$ cumplen las condiciones, por lo que tenemos $M_p(E)
		= p+1 = 26050968$.
		
		\item Para calcular el orden de $P$, nos basta con sumarlo sucesivamente consigo mismo hasta obtener
		como resultado el punto del infinito: $<0, 1, 0>$
		
		Previamente, tenemos que tener en cuenta algunos detalles:
		\begin{itemize}
			\item Si tenemos $Q = (x_q, y_q)$ y $R = (x_r, y_r)$, entonces:
			$$S = Q + R = (x_s, y_s) \equiv_p (m^2 - a - x_q - x_r, -y_q - m(x_s - x_q)) \quad \text{con }
			m=\frac{y_r - y_q}{x_r - x_q}$$
			\item Si tenemos $Q = (x, y)$ y $-Q = (x, -y)$, entonces:
			$$S = Q + (-Q) = (x_s, y_s) \equiv_p <0, 1, 0>$$
			\item Si tenemos $Q = (x, y)$, entonces:
			$$S = Q + Q = (x_s, y_s) \equiv_p (m^2 - a - 2x, -y - m(x_s - x)) \quad \text{con }
			m=\frac{3x^2 + 2ax + b}{2y}$$
		\end{itemize}
		
		Entonces, para nuestro $P = (2, 4)$ tenemos que:
		\begin{enumerate}
			\item $\displaystyle m=\frac{3x^2 + 2ax + b}{2y} = \frac{3 \cdot 2^2 + 2 \cdot 0 \cdot 2 + 4}
			{2 \cdot 4} = \frac{12 + 0 + 4}{8} = 2$
			\item $x_s \equiv_p m^2 - a - 2x = 2^2 - 0 - 2 \cdot 2 = 0$
			\item $y_s \equiv_p -y - m(x_s - x) = -4 - 2(0 - 2) = 0$
		\end{enumerate}
		
		Por tanto, se tiene que $2P = P + P = (0, 0)$.
		
		En teoría, deberíamos seguir hasta llegar al punto del infinito pero, en este caso, podemos ver
		que $2P = (0, 0)$ tiene orden 2 pues es su propio inverso. Por tanto, como $2P$ tiene orden 2,
		se tiene que $P$ tiene orden 4.
	\end{enumerate} % Noveno ejercicio
\end{document}
