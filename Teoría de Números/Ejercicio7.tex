	Sea $n$ el siguiente número primo mayor estricto que tu DNI, así $n$ tendrá 8 cifras decimales.

	\begin{enumerate}
		\item Extrae los factores potencias de 2 de $n-1$ y de $n+1$. Toma $n_1$ y $n_2$ los cofactores impares
		respectivos.
		\item Pasa el test de Miller-Rabin para varias bases y si fuera necesario el de Solovay-Strassen para ver
		si $n_1$ y $n_2$ son compuestos.
		\begin{enumerate}
			\item En caso de que hayas encontrado un certificado de composición para $n_1$ y $n_2$, aplica el método
			$\rho$ de Pollard o el de Fermat para encontrar factores de ambos.
			\item Si los factores son grandes y no has demostrado que sean compuestos, encuentra un elemento primitivo
			o una sucesión de Lucas para certificar su primalidad.
		\end{enumerate}
		\item Aplica recursivamente lo anterior hasta encontrar las factorizaciones en primos de $n-1$ y $n+1$.
	\end{enumerate}

\section*{Solución}
	\begin{enumerate}
		\item Lo primero es ver las potencias de nuestros $n-1$ y $n+1$, esto básicamente consisten en dividir
		por 2 mientras que sea par, no tiene mucha magia.
		
		Obtenemos que $n-1 = 26050966 = 2 \cdot 13025483 = 2 \cdot n_1$ y que $n+1 = 26050968 = 2^3 \cdot 3256371
		= 2^3 \cdot n_2$. Tenemos así que $n_1 = 13025483$ y que $n_2 = 3256371$.
		
		\item Para realizar el test de Miller-Rabin, es necesario comprobar si se cumple que las únicas raíces de
		la unidad en $\mathbb{Z}_n$ son 1 y -1 puesto que si $n$ es primo, $\mathbb{Z}^*_n$ es un cuerpo cíclico.
		Por tanto, como por el Teorema Pequeño de Fermat (TPF) tenemos que para cualquier elemento $a \in
		\mathbb{Z}^*_n$ se tiene que $a^{n-1} \equiv_n 1$ y sabemos que $n-1$ es par, tenemos una serie de raíces
		fáciles de calcular pues nos basta con ir dividiendo $n-1$ entre 2 hasta llegar a un $m$ impar. En cada
		paso, comprobaremos que las raíces de la unidad son 1 ó -1, si nos saliera otro valor, estaríamos seguros
		de que $n$ no es primo. El algoritmo es el siguiente\footnote{Este algoritmo es perfectamente funcional
		pero terrible en eficiencia, por lo que en la práctica se suele implementar al revés, es decir, calculando
		primero el $m$ impar y avanzando hasta $n-1$.}:
		
		\begin{algorithm}[H]
		\begin{algorithmic}[1]
			\REQUIRE \ \\
				\texttt{$n$}, número al que hacer el test \\
				\texttt{$p$}, primo con el que comprobar \\ \
			\IF{\texttt{$p^{n-1} \not\equiv_n 1$}}
				\PRINT{\texttt{No es primo}}
			\ELSE
				\STATE{\texttt{$m = \displaystyle \frac{n-1}{2}$}}
				\STATE{\texttt{$Certeza = \FALSE$}}
				\WHILE{\texttt{$m$ par \AND \NOT $Certeza$}}
					\IF{\texttt{$p^m \equiv_n -1$}}
						\PRINT{\texttt{Es primo}}
						\STATE{\texttt{$Certeza = \TRUE$}}
					\ELSIF{\texttt{$p^m \not\equiv_n 1$}}
						\PRINT{\texttt{No es primo}}
						\STATE{\texttt{$Certeza = \TRUE$}}
					\ENDIF
					\STATE{\texttt{$m = \displaystyle \frac{m}{2}$}}
				\ENDWHILE
				\IF{\texttt{$Certeza == \FALSE$}}
					\IF{\texttt{$p^m \in \{-1, 1\} \mod{n}$}}
						\PRINT{\texttt{Es primo}}
					\ELSE
						\PRINT{\texttt{No es primo}}
					\ENDIF
				\ENDIF
			\ENDIF
		\end{algorithmic}
		\caption{Test de Miller-Rabin.}
		\label{Miller-Rabin}
		\end{algorithm}
		
		Claramente, ni $n_1$ ni $n_2$ son compuestos pues fallan en la primera prueba\footnote{Nótese que deberían
		hacerse varias pruebas antes de declarar un número posible primo pero basta hacer una que nos diga que es
		compuesto para poder parar y tener seguridad de que es compuesto, como ha sido nuestro caso en el primer
		paso.}:
		\begin{itemize}
			\item $2^{n_1-1} \equiv_{n_1} 9704949 \neq 1$
			\item $2^{n_2-1} \equiv_{n_2} 2931403 \neq 1$
		\end{itemize}
		
		Llegados a este paso, quedarían dos vertientes en función de la salida del test de Miller-Rabin: compuesto
		o posible primo.
		
		\begin{enumerate}
			\item En caso de que sea compuesto, aplicaremos el método de Fermat o el $\rho$ de Pollard para
			encontrar factores de ambos.
			
			Empezaremos viendo el método de Fermat, cuyo algoritmo sería:
			\begin{algorithm}[H]
			\begin{algorithmic}[1]
				\REQUIRE \ \\
					\texttt{$n$}, número a descomponer \\ \
				\STATE{\texttt{$s = \lfloor \sqrt{n} \rfloor$}}
				\IF{\texttt{$s^2 - n == 0$}}
					\PRINT{\texttt{$n = s^2$}}
				\ELSE
					\STATE{\texttt{$s = s+1$}}
					\STATE{\texttt{$r = s^2 - n$}}
					\STATE{\texttt{$u = 2s + 1$}}
					\STATE{\texttt{$v = 1$}}
					\WHILE{\texttt{$r \neq 0$}}
						\IF{\texttt{$r > 0$}}
							\STATE{\texttt{$r = r - v$}}
							\STATE{\texttt{$v = v + 2$}}
						\ELSIF{\texttt{$r < 0$}}
							\STATE{\texttt{$r = r + u$}}
							\STATE{\texttt{$u = u + 2$}}
						\ENDIF
					\ENDWHILE
					\PRINT{\texttt{$\displaystyle n = \frac{u-v}{2} \cdot \frac{u+v-2}{2}$}}
				\ENDIF
			\end{algorithmic}
			\caption{Método de factorización de Fermat.}
			\label{Fermat-factors}
			\end{algorithm}
			
			El problema de este algoritmo salta a la vista pues necesitamos 122851 pasos para descomponer
			$n_1 = 103 \cdot 126461$ y 194 pasos para descomponer $n_2 = 1629 \cdot 1999$. Por lo que en la
			práctica no se suele usar.
			
			Sin embargo, el algoritmo $\rho$ de Pollard suele tener mejores resultados. Su definición sería la
			siguiente \footnote{Nótese que hemos usado la función $f_m: \mathbb{N} \rightarrow \mathbb{N}$
			con $z \mapsto z^2+1 \mod{m}$ aunque la función original era la dada por $z \mapsto z^2-1 \mod{m}$}:
			\begin{algorithm}[H]
			\begin{algorithmic}[1]
				\REQUIRE \ \\
					\texttt{$n$}, número a descomponer \\ \
				\STATE{\texttt{$x = 1, \quad y = 1, \quad d = 1$, valores iniciales}}
				\WHILE{\texttt{$d == 1$}}
					\STATE{\texttt{$x = f(x)$}}
					\STATE{\texttt{$y = f(f(y))$}}
					\STATE{\texttt{$d = GCD(|x-y|, n)$}}
				\ENDWHILE
				\IF{\texttt{$d == n$}}
					\PRINT{\texttt{$n$ es primo}}
				\ELSE
					\PRINT{\texttt{$\displaystyle n = d \cdot \frac{n}{d}$}}
				\ENDIF
			\end{algorithmic}
			\caption{Método de factorización $\rho$ de Pollard.}
			\label{Rho-Pollard}
			\end{algorithm}
			
			Podemos descomponer $n_1 = 103 \cdot 126461$ mediante:
			\begin{center}
			\begin{tabular}{ | r | c | c | c |}
				\hline
				i   & $x$       & $y$       & GCD \\
				\hline
				0   & 1         & 1         & 1 \\
				1   & 2         & 5         & 1 \\
				2   & 5         & 677       & 1 \\
				3   & 26        & 4424560   & 1 \\
				4   & 677       & 476428    & 1 \\
				5   & 458330    & 3436304   & 1 \\
				6   & 4424560   & 7695078   & 1 \\
				7   & 3365853   & 11721924  & 1 \\
				8   & 476428    & 9496733   & 1 \\
				9   & 1572427   & 12797088  & 1 \\
				10  & 3436304   & 9648514   & 1 \\
				11  & 11719665  & 3898809   & 1 \\
				12  & 7695078   & 11911865  & 1 \\
				13  & 1969078   & 7882229   & 1 \\
				14  & 11721924  & 5631328   & 103 \\
				\hline
			\end{tabular}
			\end{center}

			Para descomponer $n_2 = 3 \cdot 1085457$ simplemente tenemos:
			\begin{center}
			\begin{tabular}{ | r | c | c | c |}
				\hline
				i   & $x$       & $y$       & GCD \\
				\hline
				0   & 1         & 1         & 1 \\
				1   & 2         & 5         & 3 \\
				\hline
			\end{tabular}
			\end{center}

			\item En caso de que sea un posible primo, se probaría mediante los algoritmos \ref{Primarity} usando
			sucesiones de Lucas u \ref{Primitive} de búsqueda de elementos primitivos.
			
			Pero en este paso, ambos nos salían compuestos.
		\end{enumerate}
		
		\item Finalmente, tras repetir los pasos anteriores tenemos:
		\begin{itemize}
			\item $n_1 = 103 \cdot 126461$ con:
			\begin{itemize}
				\item 5 un elemento primitivo de $\mathbb{Z}^*_{103}$.
				\item 2 un elemento primitivo de $\mathbb{Z}^*_{126461}$.
				\item Ambos algoritmos pasan por los mismos números.
			\end{itemize}
			
			\item $n_2 = 3^2 \cdot 181 \cdot 1999$ con:
			\begin{itemize}
				\item 2 un elemento primitivo de $\mathbb{Z}^*_{3}$.
				\item 2 un elemento primitivo de $\mathbb{Z}^*_{181}$.
				\item 3 un elemento primitivo de $\mathbb{Z}^*_{1999}$.
				\item El algoritmo de Fermat pasa por:
				\begin{itemize}
					\item $n_2 = 1629 \cdot 1999$ en 194 pasos.
					\item $n_2 = 9 \cdot 181 \cdot 1999$ en 140 pasos.
					\item $n_2 = 3^2 \cdot 181 \cdot 1999$ en 1 paso.
				\end{itemize}
				
				\item El algoritmo $\rho$ de Pollard pasa por:
				\begin{itemize}
					\item $n_2 = 3 \cdot 1085457$ en 1 paso.
					\item $n_2 = 3^2 \cdot 361819$ en 1 paso.
					\item $n_2 = 3^2 \cdot 181 \cdot 1999$ en 14 pasos.
				\end{itemize}
			\end{itemize}
			
		\end{itemize}
	\end{enumerate}