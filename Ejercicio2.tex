	$d1$, $d2$, $d3$, $d4$, $d5$, $d6$, $d7$ y $d8$ serán los valores de los dígitos de tu DNI o pasaporte, donde
	$d1$ es tu primer dígito distinto de cero. Si te faltan dígitos, añades ceros. Así, $d1d2d3d4d5d6d7d8$ será
	un número de 8 cifras.
	
	Dado tu número de 8 cifras, $n$:
	\begin{enumerate}
		\item Halla la factorización en primos de $n$.
		\item Sea $n_1$ el mayor de los factores primos de $n$ módulo $10^5$ y $n_2$ el segundo módulo $10^4$,
		calcula las FCS de $\sqrt{n_1}$ y $\sqrt{n_2}$.
	\end{enumerate}

\section*{Solución}
	\begin{enumerate}
		\item El algoritmo que he utilizado para la descomposición en factores primos de un número dado es el
		siguiente:
		\begin{algorithm}[H]
			\begin{algorithmic}[1]
				\REQUIRE \ \\
					\texttt{$n$}, número a descomponer en primos \
				\STATE{\texttt{$lim = \sqrt{n}$}}
				\STATE{\texttt{$prime = 2$}}
				\WHILE{\texttt{$prime < lim$}}
					\WHILE{\texttt{$n \% prime = 0$}}
						\PRINT{\texttt{prime}}
						\STATE{\texttt{$\displaystyle n = \frac{n}{prime}$}}
						\STATE{\texttt{$lim = \sqrt{n}$}}
					\ENDWHILE
					\STATE{\texttt{$prime = nextPrime(n)$}}
				\ENDWHILE
				
				\IF{\texttt{$n != 1$}}
					\PRINT{\texttt{n}}
				\ENDIF
			\end{algorithmic}
			\caption{Factorización de un número dado.}
			\label{Factors}
		\end{algorithm}
		
		En particular, para nuestro $n$ obtenemos la descomposición:
		$$n = 26050919 = 17 \cdot 19 \cdot 59 \cdot 1367$$
		
		\item Para calcular los índices de la Fracción Continua Simple de un número usaremos el siguiente algoritmo:
		\begin{algorithm}[H]
			\begin{algorithmic}[1]
				\REQUIRE \ \\
					\texttt{$n$}, número a sobre el que trabajar \
				\STATE{\texttt{$q_0 = \lfloor \sqrt{n} \rfloor$}}
				\STATE{\texttt{$i = 0$}}
				\STATE{\texttt{$P_i = 0$}}
				\STATE{\texttt{$Q_i = 1$}}
				\STATE{\texttt{$q_i = q_0$}}
				\WHILE{\texttt{$q_i < 2 \cdot q_0$}}
					\STATE{\texttt{$i = i+1$}}
					\STATE{\texttt{$P_i = q_{i-1}*Q_{i-1} - P_{i-1}$}}
					\STATE{\texttt{$Q_i = (n-P_i^2) / Q_{i-1}$}}
					\STATE{\texttt{$q_i = (P_i+q_0) / Q_i$}}
					\PRINT{\texttt{$P_i, Q_i, q_i$}}
				\ENDWHILE
			\end{algorithmic}
			\caption{Algoritmo de cálculo de la FCS de la raíz de un número.}
			\label{FCS}
		\end{algorithm}
		
	\end{enumerate}
