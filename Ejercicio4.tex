	Sea $n$ el siguiente número primo\footnote{Aunque conocemos este hecho, dado que la utilidad de este ejercicio
	es conocer y poner a prueba nuestro conocimiento sobre tests de primalidad, lo ignoraremos y trataremos a $n$
	como un número que queremos ver si es primo.} mayor estricto que tu DNI, así $n$ tendrá 8 cifras decimales.
	\begin{enumerate}
		\item Con $P = 3$ y $Q = -1$, define un entero cuadrático $\alpha$ y sus sucesiones de Lucas.
		\item Para el discriminante correspondiente, $\Delta = P^2 - 4Q = 13$, calcula el símbolo de Jacobi
		$\displaystyle \left(\frac{\Delta}{n}\right)$ y define $\displaystyle r = n - \left(\frac{\Delta}{n}\right)$.
		Si tu $n$ fuera primo, ¿qué debería pasarle a $\alpha^r$? ¿Y a los términos $V_r$ y $U_r$?
		\item Calcula los términos $V_r$ y $U_r$ módulo $n$ de las sucesiones de Lucas con el algoritmo de izquierda
		a derecha.
		\item ¿Se verifica el Teorema Pequeño de Fermat (TPF) para $\alpha$? ¿Qué deduces sobre la primalidad de
		tu $n$?
	\end{enumerate}

\section*{Solución}
   \begin{enumerate}
        \item Claramente, definimos $\displaystyle \alpha = \frac{P + \sqrt{\Delta}}{2}$ con $\Delta = P^2 - 4Q$,
        y tenemos que $\Delta = 3^2 - 4(-1) = 9+4 = 13$ y $\displaystyle \alpha = \frac{3 + \sqrt{13}}{2}$
        
        Podemos comprobar que nuestro $\alpha$ es un entero algebraico pues $a = \alpha + \beta, b = \alpha
        \beta \in \mathbb{Z}$ con $\displaystyle \beta = \overline{\alpha} = \frac{P - \sqrt{\Delta}}{2}$:
        $$\displaystyle a = \frac{P + \sqrt{\Delta}}{2} + \frac{P - \sqrt{\Delta}}{2} = P = 3, \qquad
        b = \frac{P + \sqrt{\Delta}}{2} \cdot \frac{P - \sqrt{\Delta}}{2} = \frac{P^2-\Delta}{2^2} =
        \frac{P^2-(P^2-4Q)}{4} = Q = -1$$
        
        De aquí deducimos que su polinomio primitivo mínimo es $f(x) = x^2 -3x -1$. A partir de aquí, podemos
        deducir que $\alpha^2 -3\alpha -1 = 0 \Leftrightarrow \alpha^2  = 3\alpha +1 \Leftrightarrow \alpha^n =
        3\alpha^{n-1} + \alpha^{n-2}$
        
        Ahora, como $\mathbb{Q\left(\alpha\right)} = \mathbb{Q\left(\sqrt{\Delta}\right)}$ con $\left\lbrace1,
        \sqrt{\Delta}\right\rbrace$ es una base de este cuerpo, podemos escribir $\displaystyle \alpha^n =
        \frac{V_n}{2} + \frac{U_n}{2}\sqrt{\Delta}$ con $U_i, V_i \in \mathbb{Z}$.
        
        Además, podemos usar la expresión $\alpha^n  = 3\alpha^{n-1} + \alpha^{n-2}$ encontrada antes para
        concluir que $\displaystyle \frac{V_n}{2} + \frac{U_n}{2}\sqrt{\Delta} = \alpha^n  = 3\alpha^{n-1} +
        \alpha^{n-2} = 3\left(\frac{V_{n-1}}{2} + \frac{U_{n-1}}{2}\sqrt{\Delta}\right) +\left(\frac{V_{n-2}}{2} +
        \frac{U_{n-2}}{2}\sqrt{\Delta}\right) = \frac{3V_{n-1}+V_{n-2}}{2} + \frac{3U_{n-1}+U_{n-2}}{2}\sqrt{\Delta}$

        De donde concluimos que:
        $$\left\lbrace
   			V_n = 3V_{n-1}+V_{n-2} \atop
   			U_n = 3U_{n-1}+U_{n-2}
   		\right.$$
   		
   		Cumplen por tanto una relación de recurrencia, ahora sólo tenemos que calcular algunos valores para tener
   		completamente determinada la recurrencia. En nuestro caso, es fácil ver que:
		$$\left.
   			\displaystyle 1 = \alpha^0 = \frac{V_0}{2} + \frac{U_0}{2}\sqrt{\Delta} \Rightarrow V_0 = 2, \quad U_0 = 0 \atop
   			\displaystyle \frac{P + \sqrt{\Delta}}{2} = \alpha^1 = \frac{V_1}{2} + \frac{U_1}{2}\sqrt{\Delta}\Rightarrow V_1 = P = 3, \quad U_1 = 1 
   		\right.$$
   		
   		A las sucesiones de estos valores $V_n$ y $U_n$ que siguen la recurrencia que acabamos de ver es lo que
   		llamaremos \textit{Sucesiones de Lucas}.
   		
		\item Para calcular el símbolo de Jacobi $\displaystyle \left(\frac{\Delta}{n}\right)$ haremos uso del
		algorimto \ref{Jac-symbol}, dándonos el siguiente resultado:
		$$\left(\frac{\Delta}{n}\right) =\footnote{Recordemos que para este ejercicio, nuestro $n$ era
		distinto, en lugar de ser el DNI, es el siguiente primo, en mi caso, $n = 26050967$.} \left(\frac{13}
		{26050967} \right)= \left(\frac{26050967}{13}\right) = \left(\frac{7}{13}\right) = \left(\frac{13}{7}\right)
		= \left(\frac{6}{7}\right) = \left(\frac{2}{7}\right) \cdot \left(\frac{3}{7}\right) = 1 \cdot
		\left(\frac{3}{7}\right) = -\left(\frac{7}{3}\right) = -\left(\frac{1}{3}\right) = -1$$
		
		De aquí, deducimos inmediatamente que $\displaystyle r = n - \left(\frac{\Delta}{n}\right) = 26050968$
		
		Ahora, estudiaremos los valores teóricos de $\alpha^r$, $U_r$ y $V_r$ en el caso de que $n$ fuese primo.
		Si $n$ es primo, tenemos que $\displaystyle |m| < |n| \Rightarrow \gcd(m, n) = 1 \quad \forall m \in
		\mathbb{Q}$ y podemos aplicar un resultado que se deduce del Pequeño Teorema de Fermat(PTF)\footnote{Hay
		un resultado análogo para $V_r$ y $U_r$ que básicamente dice que $U_r \equiv_n 0$ y que $V_r \equiv_n
		2\alpha^r$.}. En particular, tenemos que $|2Q\Delta| < |n|$ por lo que se cumple dicho resultado:
		$$\alpha^r = \alpha^{n-\left(\frac{\Delta}{n}\right)} \equiv_n \left\lbrace
			1 \qquad \text{Si } \left(\frac{\Delta}{n}\right) = 1 \atop
			Q \qquad \text{Si } \left(\frac{\Delta}{n}\right) = -1
		\right.$$
		
		Por lo que para nuestro caso particular se tiene que $\alpha^r \equiv_n Q = -1$. Trivialmente se deduce
		que $U_r \equiv_n 0$ y $V_r \equiv_n 2Q = -2$
		
		\item Para este apartado, reutilizaremos el algoritmo \ref{Fast-exp} con pequeñas modificaciones pues,
		aunque sirve para calcular $\alpha^r$, no es tan fácil calcular $U_r$ y $V_r$ a partir de él.
		Principalmente, tenemos que modificar la salida para que nos devuelva por separado la componente racional
		de la irracional, tener en cuenta que $V_i$ y $U_i$ son el doble de las componentes racional e irracional
		respectivamente y redefinir el producto aprovechándonos de la siguiente propiedad:
		$$\left(a+b\sqrt{e}\right)\left(c+d\sqrt{e}\right)=\left(ac+bde\right)+\left(ad+cb\right)\sqrt{e}$$
		
		Finalmente, el algoritmo quedaría:
		\begin{algorithm}[H]
			\begin{algorithmic}[1]
				\REQUIRE \footnote \ \\
					\texttt{$a_r$}, base racional \\
					\texttt{$a_i$}, base irracional \\
					\texttt{$b =_b e_je_{j-1}e_{j-2}\dots e_2e_1e_0$}, exponente y su expresión en binario \\
					\texttt{$c$}, módulo\\ \
				\STATE{\texttt{$acu_r = 1$}}
				\STATE{\texttt{$acu_i = 0$}}
				\STATE{\texttt{$exp = 0$}}
				\STATE{\texttt{$i = j$; $j$ es el número de dígitos en binario de $b$}}
				\WHILE{\texttt{$i > 0$}}
					\STATE{\texttt{$(acu_r, acu_i) \equiv (acu_r, acu_i)^2 \mod{c}$ }}
					\STATE{\texttt{$exp = 2 \cdot exp$}}
					\IF{\texttt{$e_i = 1$}}
						\STATE{\texttt{$(acu_r, acu_i) \equiv (acu_r, acu_i) \cdot (a_r, a_i) \mod{c}$}}
						\STATE{\texttt{$exp = exp + 1$}}
					\ENDIF
					\STATE{\texttt{$i = i - 1$}}
				\ENDWHILE
				\PRINT{\texttt{$2 \cdot (acu_r, acu_i) \mod{c}$, $exp$}}
			\end{algorithmic}
			\caption{Exponencialización rápida de irracionales de izquierda a derecha}
			\label{Fast-exp-irr}
		\end{algorithm}

		\footnotetext{Nótese que las bases que se le pasan como entrada deben ser los valores $\displaystyle
		\frac{V_1}{2}$ y $\displaystyle \frac{U_1}{2}$ módulo $n$ respectivamente.}
		
		Los resultados obtenidos son:
		\begin{center}
		\begin{tabular}{ | r | c | c |}
			\hline
			i           & $V_i$     & $U_i$ \\
			\hline
			1           & 3         & 1 \\
			3           & 36        & 10 \\
			6           & 1298      & 360 \\
			12          & 1684802   & 467280 \\
			24          & 18363915  & 14055820 \\
			49          & 11672958  & 4428956 \\
			99          & 20937720  & 23911249 \\
			198         & 23382471  & 21523686 \\
			397         & 15848977  & 22985641 \\
			795         & 25678459  & 18739467 \\
			1590        & 14759824  & 15744084 \\
			3180        & 20679959  & 20945882 \\
			6360        & 6271843   & 2028203 \\
			12720       & 18332426  & 13856864 \\
			25440       & 22940247  & 25097842 \\
			50880       & 25379149  & 18394763 \\
			101761      & 463429    & 21134127 \\
			203523      & 14461565  & 17713358 \\
			407046      & 11784831  & 20336956 \\
			814092      & 3753301   & 21713753 \\
			1628185     & 5185797   & 9148345 \\
			3256371     & 11511884  & 520085 \\
			6512742     & 15819230  & 20750332 \\
			13025484    & 0         & 2635672 \\
			26050968    & 26050965  & 0 \\
			\hline
		\end{tabular}
		\end{center}
		
		\item Claramente, nuestro $n$ satisface los resultados teóricos que se esperan de un número primo
		pero eso no nos garantiza que sea primo. El Pequeño Teorema de Fermat(PTF) y el resultado que se
		deduce del mismo sólo nos permiten comprobar si el número es compuesto cuando no se verifican los
		resultados teóricos esperados.
    \end{enumerate}
