	Sea $n$ el siguiente número primo\footnote{Aunque conocemos este hecho, dado que la utilidad de este ejercicio
	es conocer y poner a prueba nuestro conocimiento sobre tests de primalidad, lo ignoraremos y trataremos a $n$
	como un número que queremos ver si es primo.} mayor estricto que tu DNI, así $n$ tendrá 8 cifras decimales.
	\begin{enumerate}
		\item Con $P = 3$ y $Q = -1$, define un entero cuadrático $\alpha$ y sus sucesiones de Lucas.
		\item Para el discriminante correspondiente, $\Delta = P^2 - 4 \cdot Q = 13$, calcula el símbolo de Jacobi
		$\displaystyle \left(\frac{\Delta}{n}\right)$ y define $\displaystyle r = n - \left(\frac{\Delta}{n}\right)$.
		Si tu $n$ fuera primo, ¿qué debería pasarle a $\alpha^r$? ¿Y a los términos $V_r$ y $U_r$?
		\item Calcula los términos $V_r$ y $U_r$ módulo $n$ de las sucesiones de Lucas con el algoritmo de izquierda
		a derecha.
		\item ¿Se verifica el Teorema Pequeño de Fermat (TPF) para $\alpha$? ¿Qué deduces sobre la primalidad de
		tu $n$?
	\end{enumerate}

\section*{Solución}
