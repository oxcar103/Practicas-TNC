	Sea $n$ el siguiente número primo\footnote{Aunque conocemos este hecho, dado que la utilidad de este ejercicio
	es conocer y poner a prueba nuestro conocimiento sobre tests de primalidad, lo ignoraremos y trataremos a $n$
	como un número que queremos ver si es primo.} mayor estricto que tu DNI, así $n$ tendrá 8 cifras decimales.
	\begin{enumerate}
		\item Con $P = 3$ y $Q = -1$, define un entero cuadrático $\alpha$ y sus sucesiones de Lucas.
		\item Para el discriminante correspondiente, $\Delta = P^2 - 4Q = 13$, calcula el símbolo de Jacobi
		$\displaystyle \left(\frac{\Delta}{n}\right)$ y define $\displaystyle r = n - \left(\frac{\Delta}{n}\right)$.
		Si tu $n$ fuera primo, ¿qué debería pasarle a $\alpha^r$? ¿Y a los términos $V_r$ y $U_r$?
		\item Calcula los términos $V_r$ y $U_r$ módulo $n$ de las sucesiones de Lucas con el algoritmo de izquierda
		a derecha.
		\item ¿Se verifica el Teorema Pequeño de Fermat (TPF) para $\alpha$? ¿Qué deduces sobre la primalidad de
		tu $n$?
	\end{enumerate}

\section*{Solución}
   \begin{enumerate}
        \item Claramente, definimos $\displaystyle \alpha = \frac{P + \sqrt{\Delta}}{2}$ con $\Delta = P^2 - 4Q$,
        y tenemos que $\Delta = 3^2 - 4(-1) = 9+4 = 13$ y $\displaystyle \alpha = \frac{3 + \sqrt{13}}{2}$
        
        Podemos comprobar que nuestro $\alpha$ es un entero algebraico pues $a = \alpha + \beta, b = \alpha
        \beta \in \mathbb{Z}$ con $\displaystyle \beta = \overline{\alpha} = \frac{P - \sqrt{\Delta}}{2}$:
        $$\displaystyle a = \frac{P + \sqrt{\Delta}}{2} + \frac{P - \sqrt{\Delta}}{2} = P = 3, \qquad
        b = \frac{P + \sqrt{\Delta}}{2} \cdot \frac{P - \sqrt{\Delta}}{2} = \frac{P^2-\Delta}{2^2} =
        \frac{P^2-(P^2-4Q)}{4} = Q = -1$$
        
        De aquí deducimos que su polinomio primitivo mínimo es $f(x) = x^2 -3x -1$. A partir de aquí, podemos
        deducir que $\alpha^2 -3\alpha -1 = 0 \Leftrightarrow \alpha^2  = 3\alpha +1 \Leftrightarrow \alpha^n  =
        3\alpha^{n-1} + \alpha^{n-2}$
        
        Ahora, como $\mathbb{Q\left(\alpha\right)} = \mathbb{Q\left(\sqrt{\Delta}\right)}$ con $\left\lbrace1,
        \sqrt{\Delta}\right\rbrace$ es una base de este cuerpo, podemos escribir $\displaystyle \alpha^n =
        \frac{V_n}{2} + \frac{U_n}{2}\sqrt{\Delta}$ con $U_i, V_i \in \mathbb{Z}$.
        
        Además, podemos usar la expresión $\alpha^n  = 3\alpha^{n-1} + \alpha^{n-2}$ encontrada antes para
        concluir que $\displaystyle \frac{V_n}{2} + \frac{U_n}{2}\sqrt{\Delta} = \alpha^n  = 3\alpha^{n-1} +
        \alpha^{n-2} = 3\left(\frac{V_{n-1}}{2} + \frac{U_{n-1}}{2}\sqrt{\Delta}\right) -\left(\frac{V_{n-2}}{2} +
        \frac{U_{n-2}}{2}\sqrt{\Delta}\right) = \frac{3V_{n-1}-V_{n-2}}{2} + \frac{3U_{n-1}-U_{n-2}}{2}\sqrt{\Delta}$

        De donde concluimos que:
        $$\left\lbrace
   			V_n = 3V_{n-1}-V_{n-2} \atop
   			U_n = 3U_{n-1}-U_{n-2}
   		\right.$$
   		
   		Cumplen por tanto una relación de recurrencia, ahora sólo tenemos que calcular algunos valores para tener
   		completamente determinada la recurrencia. En nuestro caso, es fácil ver que:
		$$\left.
   			\displaystyle 1 = \alpha^0 = \frac{V_0}{2} + \frac{U_0}{2}\sqrt{\Delta} \Rightarrow V_0 = 2, \quad U_0 = 0 \atop
   			\displaystyle \frac{P + \sqrt{\Delta}}{2} = \alpha^1 = \frac{V_1}{2} + \frac{U_1}{2}\sqrt{\Delta}\Rightarrow V_1 = P = 3, \quad U_1 = 1 
   		\right.$$
   		
   		A las sucesiones de estos valores $V_n$ y $U_n$ que siguen la recurrencia que acabamos de ver es lo que
   		llamaremos \textit{Sucesiones de Lucas}.
    \end{enumerate}

